\chapter{Related Work}
\label{ch:related-work}

Because of Svelte's relative novelty, not much scientific work has been published about the language. Nonetheless, we found two works comparing Svelte to other contemporary JavaScript frameworks. \todo{Some more...}

\section{Angular and Svelte Frameworks: a Comparative Analysis}

\todo{Don't like this paper...}

\section{DOM benchmark comparison of the front-end JavaScript frameworks React, Angular, Vue, and Svelte}
In their work Levlin conducted a performance analysis comparing React\footnote{\url{https://react.dev/}}, Angular\footnote{\url{https://angular.io/}}, Vue\footnote{\url{https://vuejs.org/}}, and Svelte \cite{levlin_dom_2020}. These performance benchmarks primarily focused on DOM manipulations and compared the frameworks speed when inserting, editing, and deleting DOM nodes. Levlin's results show that Svelte is among the fastest of the compared frameworks when manipulating only few nodes. On the other hand, Svelte struggles when manipulating large amounts of elements.    
\todo{finish...}

% \begin{itemize}
%     \item Levlin\cite{levlin_dom_2020}
%     \item comparative performance analysis between, React, Angular, Vue, Svelte.
%     \item Performance key takeaway. React, Vue and Angular, Svelte, behave similar because of (not) using virtual DOM.
%     \item React winner for mass updates, svelte winner for single updates
%     \item BUT, benchmark for mass updates seem disingenuous, because Svelte benchmark does not use Svelte feature and instead directly manipulates DOM. 
% \end{itemize}

\section{State of JavaScript}
State of JavaScript is a yearly Survey run by S. Greif and E. Burel \cite{greif_state_2022}. The survey intends to identify trends in the web development. Therefore, helping developers when making technology choices. It asks questions about JavaScript language features, frontend-frameworks, meta-frameworks, and build tools. As an openly accessible survey, it can not guarantee to be representative for the entire developer ecosystem. Instead, it shows results for a subset of developers.

The results for 2022 show Svelte as the fourth most used frontend framework with 21.1\%, behind React (81.8\%), Angular (48.8\%), and Vue (46.2\%). In terms of retention (participants that would use the framework again), Svelte placed second (89.7\%), behind Solid (90.9\%). SvelteKit also ranked fourth in terms of usage (11.9\%) behind Next.js\footnote{\url{https://nextjs.org/}} (48.6\%), Gatsby\footnote{\url{https://www.gatsbyjs.com/}} (23\%), and Nuxt\footnote{\url{https://nuxt.com/}} (18.1\%). SvelteKit too placed second in for retention (\todo{percent}), behind Astro (92.8\%).    
\todo{finish...}



% \begin{itemize}
%     \item \cite{noauthor_state_nodate}
% \end{itemize}
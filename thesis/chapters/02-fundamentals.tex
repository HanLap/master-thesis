\chapter{Fundamentals}
\label{ch:fundamentals}


% todo: React Server Components

This Chapter introduces fundamental knowledge that will be required in the following thesis.


\section{Methodology}
\label{sec:methodology}

This thesis follows the principles of an explorative case study as defined by P. Runeson and M. Höst\cite{runeson_guidelines_2009}. The Authors describe the Process as consisting of five steps:

\begin{enumerate}
    \item Case study Design: Objectives are defined, and the case study is planned.
    \item Preparation for data collection: procedures and protocols for data collection are defined.
    \item Collecting evidence: execution with data collection on the studied case.
    \item Analysis of the collected data.
    \item Reporting
\end{enumerate}

As defined in \ref{sec:purpose-of-this-study}, this study intents to provide evidence that points towards or against SvelteKit's applicability for the development of business applications.

\subsection{Definition of Data Collection}
\begin{enumerate}
    \item Performance metrics
    \item complexity metrics
\end{enumerate}

\subsection{Collection of evidence}

Metrics applied to all implementations


\subsection{Evaluation}

Factual evaluation:
\begin{enumerate}
    \item LoC
    \item performance
    \item ecosystem size
\end{enumerate}

\section{Programmer Experience}


\section{Web Development}
\label{sec:web-development}

\subsection{Rendering Patterns}
\label{sec:rendering-patterns}

\begin{itemize}
    \item At first only static files
    \item later SSR (PHP and stuff)
    \item CSR with dominance of SPA's heralded by React
    \item now step back to hybrid approach using SSR or SSG with client side hydration.
\end{itemize}

\begin{figure}
    \centering
    \begin{sequencediagram}
      \newthread{A}{Client}{}
      \newinst[1]{B}{Server}{}
      \newinst[1]{C}{Database}{}
      \begin{call}{A}{vist /auftraege}{B}{index.html}
      \end{call}
      \begin{call}{A}{request Data}{B}{}
        \begin{call}{B}{SELECT *}{C}{}
        \end{call}
      \end{call}
      \begin{callself}{A}{render data}{}
      \end{callself}
    \end{sequencediagram}

    \label{fig:timing-spa}
    \caption{Example load timing for an application using client-side rendering.}
\end{figure}

\subsection{Meta Frameworks}

\section{Svelte}
\label{sec:svelte}

% svelte is component library
% created by Rich Harris as successor to Ractive.js \cite{offerzen_origins_svelte_2022}
% 
% language for developing user interfaces on the web \cite{offerzen_origins_svelte_2022}
% 
% framework not something to run code but to think about code
% 
% tries to do as much as possible at build-time instead of runtime
% 
% compiler means it is possible to do things which would not work in vanilla JS

Svelte was created in 2016 by Rich Harris as a successor to Ractive.js \cite{offerzen_origins_svelte_2022}. It is a framework for developing web user interfaces, similar to React, Angular, and Vue.js. Harris describes Svelte as a framework not to run code but to think about code \cite{offerzen_origins_svelte_2022}. Svelte's key difference compared to other web UI frameworks is, that it uses a Compiler. This makes it possible to shift many rendering steps from the run time into the build time. Therefore, Svelte can ship smaller and more efficient bundles. Furthermore, Svelte's Syntax can do things which would not be possible in vanilla JavaScript.

\subsection{Features}

\subsubsection{Svelte Files}
% - Files with .svelte ending
% - Syntax very similar to regular html with some extensions
% - standard javascript/Typescript $=>$ existing tools can be more easily integrated
% - javascript/html/css colocated in same file
% - {} to use js expressions in html, similar to react.
% - shorthand attribute syntax

Svelte is a custom language that is very similar to HTML, JavaScript, and CSS, which is colocated inside a \texttt{.svelte} file. In fact, plain HTML syntax is also valid Svelte syntax:

\begin{myminted}{svelte}{app.svelte}
<h1>Hello World!</h1> 
\end{myminted}

Logic can be added with a \texttt{<script>}-block and variables can be used outside the HTML with curly-braces:

\begin{myminted}{svelte}{}
<script>
    const name = 'world';
</script>

<div>Hello {name.toUpperCase()}!</div>
\end{myminted}

Similar to React, variables can also be used in element attributes, and Svelte even provides a shorthand when variable name and attribute name are the same:
\begin{myminted}{svelte}{}
<script>
    const src = './some/img.svg';
</script>

<img src={src}>
<img {src}> 
\end{myminted}

% ToDo: write about TypeScript support.

\subsubsection{Reactivity}
% - asignments with usual js asignments. would not work in React.
% - unlike react, svelte runs script just once, for assignments, svelte inserts invalidate blocks to update a variable
% - compare svelte to equivalent react code.

Svelte provides a strong system for reactivity. Mutation of variables is done with regular assignment operations:

\begin{myminted}{svelte}{}
<script>
    let count = 0;

    function increment() {
        count += 1;
    }
</script>

<div>count: {count}!</div>
<button on:click={increment}>click me!</button>
\end{myminted}

This is possible, because Svelte's compiler replaces assignment operations at build time. The increment function in the prior example would be compiled to this statement:

\begin{myminted}[escapeinside=||]{js}{}
function increment() {
    |\$\$|invalidate('count', count += 1)
}
\end{myminted}

An equivalent React implementation using function components would have to use more verbose semantics, because without a specialized compiler, assignment operations do not have any special meaning:

\begin{myminted}{jsx}{}
function Component() {
    const [count, setCount] = useState(0);

    function increment() {
        setCount(c => c + 1);
    }

    return <>
        <div>count: {count}!</div>
        <button onClick={increment}>click me!</button>
    </>
}
\end{myminted}

React's code runs every time, a component is updated, this means derived values can just be declared, and they are evaluated every time. Svelte's code on the other hand, only runs once on component initialization. Therefore, Svelte introduces custom syntax for defining reactive statements. To This end, Svelte repurposes labeled statements:


\begin{myminted}[highlightlines={4},escapeinside=||]{svelte}{}
<script>
    let count = 2;

    |\$|: doubled = count * 2;

    function increment() {
        count += 1;
    }
</script>

<button on:click="{increment}">click me!</button>
<div>count: {count}!</div>
<div>doubled: {doubled}!</div>
\end{myminted}

% - $: to make reactive statement
% - repurpose of rarely used js feature
% - used for reactive assignments, conditions, and whole codeblocks
% - reactive block is rerun everytime a variable used inside the block changes

\mintinline{js}{$:} marks a statement as reactive. Every time a value, that is used in a reactive statement, changes, the statement is reevaluated. This can be used for assignments, as seen in the prior example, but it can also be used for conditions and code blocks:

\begin{myminted}[escapeinside=||]{svelte}{}
<script>
    let count = 0;

    |\$|: {
        console.log(`new value of count: |\$|{count}`);
        console.log('This will also be executed when count changes');
    }

    |\$|: if(count > 9) {
        console.log('It is over 9!');
        count = 0;
    }
</script>
\end{myminted}

\subsubsection{Templates}

% - Templates for if, each

HTML does not have a way of expressing logic, thus Svelte introduces its own syntax. To conditionally render some markup, it is wrapped inside an if-block:

\begin{myminted}[highlightlines={11-14}]{svelte}{}
<script>
    let count = 0;

    function increment() {
        count += 1;
    }
</script>

<button on:click={increment}>click me!</button>
<div>count: {count}!</div>
{#if count > 9}
    <div>Vegeta what does the scouter say about his power level?</div>
    <div>It's over 9!!</div>
{:else}
    <div>The number ir pretty low</div>
{/if}
\end{myminted}

UI's also often work with lists of data. To this end, Svelte provides an each-block, which can be used to handle repeating data:

\begin{myminted}[highlightlines={5-7}]{svelte}{}
<script>
    let array = [1, 2, 3, 4];
</script>

{#each array as value}
    <div>{value}</div>
{/each}
\end{myminted}

\subsubsection{Components}

% - Since emergence of React, component driven development became norm.
% - in Svelte, one file = one component.
% - Name = Component Name.

% todo: citation needed
Since the emergence of React in 2013, component-driven user-interfaces have become a widespread approach to web development. In Svelte, a component is represented by a singular \mintinline{html}|.svelte|-file, where the file name determines the component name: 

\begin{myminted}{html}{MyComponent.svelte}
<script>
    export let value;
</script>

<div>hello, {value}!</div>
\end{myminted}

To define props on a Component, Svelte repurposes the \mintinline{js}|export| keyword. The Component can than be imported and used like a regular element:

\begin{myminted}{svelte}{app.svelte}
<script>
    import MyComponent from './MyComponent.svelte';

    let value = 'world';
</script>

<MyComponent {value} />
\end{myminted}

\subsubsection{Events}

% - possible to listen to any event with on: 
% - custom events also

In a prior example we already showed usage of \mintinline{svelte}|on:click|. This directive is used to listen to click events. Svelte make sit possible to listen for arbitrary events on an element with the \mintinline{html}|on|-keyword:

\begin{myminted}[highlightlines={7}]{html}{}
<script>
    function handleMove(event) {
        // ...
    }
</script>

<div on:pointermove="{handleMove}" ></div>
\end{myminted}

Furthermore, Svelte provides an event dispatcher, so that components can throw their own custom events:

\begin{myminted}{html}{}
<script>
    import { createEventDispatcher } from 'svelte';

    const dispatch = createEventDispatcher();

    function sayHello() {
        dispatch('message', {
            text: 'Hello!'
        });
    }
</script>
\end{myminted}

\subsubsection{Bindings}

% - usually data passed down. Parent can set props of child but not otherway around.
% - Svelte provides tool to work around that
% - simpler compared to react

\begin{myminted}{svelte}{}
<script>
    let name = 'Peter';
</script>

<input bind:value={name} />
\end{myminted}

\begin{myminted}{jsx}{}
function Component() {
    const [name, setName] = useState('Peter');

    return ( 
        <input 
            value={name}
            onChange={e => setName(e.currentTarget.value)} 
        />
    )
}
\end{myminted}

\section{SvelteKit}
\label{sec:sveltekit}

% Meta Framework
% Stable release 2022
% provides functionality for modern applications
% Deployable as static files, web server, edge functions

\subsection{Features}

\subsubsection{Loading Data}
\label{sec:sveltekit-loading}

Svelte provides standard way for loading data.

\subsubsection{Routing}

\subsubsection{Server Actions}
\label{sec:sveltekit-server-actions}
 
\subsubsection{Rendering}

\subsubsection{Progressive Enhancement}

\section{Business Application}
\label{sec:business-application}

Our applications usually not public facing $\rightarrow$ SEO not relevant $\rightarrow$ SSR, FCP, TTI not as relevant

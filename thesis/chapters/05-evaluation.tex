\chapter{Evaluation}
\label{ch:evaluation}
In this chapter we go over our evaluation results. We discuss the outcome of our performance benchmarks. Afterwards, we compare SvelteKit and UI5 on the basis of multiple different code examples to find out how they perform in terms of developer experience. Finally, we discuss further notes and findings about Svelte and SvelteKit we uncovered during our study.

\section{Performance}
\label{sec:evaluation-performance}
To gain insights into SvelteKit's performance, we ran benchmarks on the existing UI5 implementation as well as all newly created SvelteKit implementations. Levlin already tested Svelte's performance in DOM manipulation tasks \cite{levlin_dom_2020}, thus we were primarily interested how SvelteKit would fare when loading pages as well as navigating between pages. To this end, we intended to measure the time to first byte (TTFB), first contentful paint (FCP), largest contentful paint (LCP), and time until a new view has rendered after navigation. Following the definitions of web.dev \cite{noauthor_webdev_nodate}, TTFB describes the time it takes until the first byte of data reaches the browser, after first loading a website. First contentful paint describes the time it takes until the browser first renders part of the application to the screen, while largest contentful paint describes the time it takes until the largest image or text block visible within the viewport is rendered. Ordinarily, Lighthouse, a tool for measuring website performance integrated in Chromium, would have been used to measure TTFB, FCP, and LCP. But, we realized that Lighthouse's automated measurement of LCP did not provide a realistic measurement for how long the UI5 implementation took to render its content. Therefore, we resorted to using Chromium's performance tool to create performance recordings. We then took measurements of LCP manually from the recording by checking when all content has been rendered to the screen. The detailed benchmarking process is documented in \Cref{app:benchmark-setup}.  

All measurements were conducted on a Lenovo ThinkPad T480 with an Intel Core i7-8550U and 16 GB RAM running Windows 11. The measurements where taken using the developer tools of Google Chrome 116.0. We benchmarked the existing UI5 implementation, the SvelteKit full stack implementation (SK-FS), and the SvelteKit implementation, using SvelteKit primarily as a frontend framework, but redirecting requests through the SvelteKit server (SK-FE). Furthermore, we were interested how SvelteKit would fare when deploying it as static files in form of a classic SPA (SK-SPA). 

\begin{figure}
    \centering
    \begin{tikzpicture}
        \begin{axis}[
                ybar,
                bar width=.5cm,
                width=.9\textwidth,
                height=.5\textwidth,
                legend style={at={(0,1)},
                    anchor=north west,legend columns=1},
                legend cell align={left},
                symbolic x coords={TTFB,FCP,LCP,Nav},
                xtick=data,
                enlarge x limits={0.15},
                nodes near coords,
                nodes near coords align={vertical},
                node near coords style={rotate=60,xshift=0.4cm,yshift=-0.2cm},
                ymin=0,ymax=4000,
                ylabel={duration (ms)},
            ]
            \addplot table[x=type,y=ui5]{\benchmarkdata};
            \addplot table[x=type,y=sk-fs]{\benchmarkdata};
            \addplot table[x=type,y=sk-fe]{\benchmarkdata};
            \addplot table[x=type,y=sk-spa]{\benchmarkdata};
            \legend{UI5, SvelteKit Full Stack, SveleKit Frontend, SvelteKit SPA}
        \end{axis}
    \end{tikzpicture}
    \caption{Benchmark Results for TTFB, FCP, LCP and Nav}
    \label{fig:benchmark}
\end{figure}

The benchmark results shown in \Cref{fig:benchmark} mostly confirm our expectations that SvelteKit would perform better in the LCP and navigation metrics. The results of SK-SPA are particularly interesting. As expected by forgoing SSR, it shows the fastest TTFB of all three SvelteKit implementations. But it takes slightly longer than the other two implementations to render its content. This confirms the assumption that SSR improves FCP outlined in \Cref{sec:rendering-patterns}. But despite its fast TTFB, the SPA implementation still has the slowest FCP of all implementations. This is a result of how SvelteKit loads its data. The UI5 implementation shows a loading spinner immediately after its FCP, before fetching data from the backend. The SK-SPA implementation on the other hand first awaits the results of all load functions relevant to the current page before rendering anything to the screen. While this approach makes sense for a server rendered application, if the application is being rendered in the browser exclusively, this can mean a prolonged blank screen before data is shown. Nonetheless, in our implementation the load times proved to be fast enough to barely matter. 

The Benchmarks also show that all SvelteKit implementations perform faster when navigating, compared to the UI5 implementation. Furthermore, SvelteKit provides out of the box support for preloading all required code and data for a page as soon as the user hovers over a link to that page. This can make page loads appear instantaneous to a user in some cases. For our benchmarks we disabled this feature to make comparisons more meaningful.

Overall, our performance analysis shows that SvelteKit can significantly improve initial page load times compared to UI5. This metric is most relevant for conversion rate \cite{noauthor_load_nodate}, a metric for public facing-pages, such as e-commerce platforms. Nonetheless, fast page load times can contribute to the overall user experience for all types of applications. 

\section{Development Comparison}
In this chapter we compare SvelteKit and UI5 in terms of ergonomics and developer experience. To this end, we show implementations of common use-cases for both frameworks.

\subsection{Page Definitions}
% mvc vs one component
% Svelte file vs hipster HTML
  % separate system from HTML,
  % resources for HTML dont help
  % new apis have to be supported first
% UI5 older roots = bagage e.g. import statements

UI5 follows the model-view-controller (MVC) design pattern. A component consists of two files. The view file is usually written in XML and describes the UI layout. The behavior is defined using a controller file usually written in JavaScript. In Svelte on the other hand, layout and behavior is written in a single Svelte file (\Cref{sec:svelte-file}). 

In UI5, the layout is written by nesting UI components in an XML file. It is not possible to use HTML directly, instead all HTML functionality has to be somehow handled by a UI5 component. In contrast, Svelte augments HTML with custom components and special syntax. This approach makes it possible to fall back to HTML in cases where Svelte has not support for the given functionality. UI5 on the other hand, has to explicitly support every HTML feature through its components. Furthermore, knowledge resources about HTML are of limited use for UI5 development.

UI5 started development in 2008, thus the framework made design decisions for features such as imports long before they were standardized. As a result, import statements in UI5 feel outdated from a modern perspective (\Cref{fig:evaluation-ui5-import}) and do not integrate nicely with libraries that use the standard ESM syntax\footnote{\url{https://blogs.sap.com/2017/04/30/how-to-include-third-party-libraries-modules-in-sapui5/}}. On the other hand, imports in Svelte follow the ESM syntax and thus integrates nicely with most modern libraries.


\begin{listing}[H]
\begin{myminted}{js}{}
sap.ui.define([
  "sap/ui/core/Control",
  "sap/m/RatingIndicator",
  "sap/m/Label",
  "sap/m/Button"
], (Control, RatingIndicator, Label, Button) => {
  // ...
});
\end{myminted}
\caption{Example of UI5's import syntax.}
\label{fig:evaluation-ui5-import}
\end{listing}

\subsection{Components}
Creation of custom components is an essential feature in many modern UI Frameworks, It enables composition and reuse of isolated UI pieces such as buttons and inputs. As noted in \Cref{sec:implementation-ui}, we initially tried to use UI5 web components as a component library in Svelte. But, after realizing multiple shortcomings with this approach, we pivoted to writing our own component library in Svelte using a predefined CSS style library. This provided multiple insights into Svelte's applicability for creating custom UI components.

In the following we show an example component that provides a button which triggers an alert when clicked. The alert shows a custom text that is passed to the control. \Cref{fig:evaluation-component-svelte} shows a possible implementation in Svelte for such a component. The implementation is 7 lines of code long. \Cref{fig:evaluation-component-ui5}, shows a corresponding implementation in UI5. With 19 lines of code, this implementation is immediately longer. Furthermore, because the code is split across two files, it is harder to understand its behavior in a single read. 

\begin{listing}[H]
\begin{myminted}{svelte}{AlertButton.svelte}
<script>
  export let text;

  function onShowAlert() {
    alert(text);
  }
</script>

<button on:click={onShowAlert}>Show Alert</button>
\end{myminted}
\caption{Alert button implementation in Svelte.}
\label{fig:evaluation-component-svelte}
\end{listing}

\begin{listing}[H]
\begin{myminted}{xml}{AlertButton.control.xml}
<core:FragmentDefinition 
    xmlns="sap.m" 
    xmlns:core="sap.ui.core" 
    xmlns:layout="sap.ui.layout">
  <Button text="Show Alert" press="onShowAlert" />
</core:FragmentDefinition>
\end{myminted}
\begin{myminted}{js}{AlertButton.js}
sap.ui.define(
  ['sap/ui/core/XMLComposite'], 
  (XMLComposite) => {
  "use strict";

  return XMLComposite.extend("ui5.example.control.AlertButton", {
    metadata: {
      properties: { text: {type : "string" } }
    },

    onShowAlert() {
      alert(this.getText());
    },
  });
});
\end{myminted}
\caption{Alert button implementation in UI5.}
\label{fig:evaluation-component-ui5}
\end{listing}

On the other hand the UI5 implementation is stronger, because it uses a UI5 component per default. The UI5 component is styled following the Fiori Design guideline, follows accessibility guidelines and ensures correct keyboard navigation behavior. In this regard UI5 can still save time, even tough the initial implementation code is more verbose. Nonetheless, the shown example is very simple. In complex use cases where UI5 does not provide prebuilt functionality, Svelte will likely pull ahead in terms of usability.
 

\subsection{Routing}
SvelteKit decided to integrate a filesystem-based routing system where files and directories in the file system have special meaning (\Cref{sec:sveltekit-routing}). UI5 on the other hand uses a declarative routing approach. Routes are declared and linked to a view in a manifest file (\Cref{fig:evaluation-ui5-routing}). 

SvelteKit's approach comes with some inherent advantages and disadvantages. Probably the biggest advantages is that it is immediately clear what the purpose of a file is. Especially for simple cases this routing system works with no overhead and makes it possible to work very fast when adding new routes. But projects with a large amount of routes will inevitably have many files with the same name. In our experience this could make it harder to use IDE search functions to quickly navigate between many files because searching most of the time returned more than one \mintinline{text}|+page.svelte| or \mintinline{text}{+page.js} file. With UI5's approach, the route definitions are always found in the same place. While this means that the purpose of a file is not always immediately deducible, it is still easy enough to look up where a file is used. UI5's approach also provides more flexibility in structuring the project.

Another effect of SvelteKit's approach is that solutions to some problems can cause a lot of file changes in a version control system. During our implementation efforts, we found the need to wrap multiple routes in a layout group. This meant, all files hat to be moved to a new subdirectory. This caused a lot of noise in the version control system for a change that was simple in nature. In UI5 the same problem could have been solved by simple changing the router configuration.

It has also to be noted that this filesystem-based approach to routing makes it effectively impossible to split a singular SvelteKit application into multiple smaller projects. Because routing is strictly bound to the filesytem, it is only possible to configure routing in the root project.

\begin{listing}[H]
\begin{myminted}{json}{manifest.json}
{
  "sap.ui5": {
    "routing": {
      "routes": [
        {
          "pattern": "",
          "name": "home",
          "target": "home"
        },
        {
          "pattern": "products/{productId}",
          "name": "productDetails",
          "target": "productDetails"
        }
      ],
      "targets": {
        "home": {
          "viewName": "Home",
        },
        "productDetails": {
          "viewName": "product/detail/ProducDetails"
        }
      }
    }
  }
}
\end{myminted}
\caption{Example routing configuration for a UI5 application with a Home view and a detail view for products.}
\label{fig:evaluation-ui5-routing}
\end{listing}

\section{Findings}
In this section we discuss further aspects of SvelteKit we discovered during our study which were not easily comparable to UI5. These finding shall server as further help when deciding if SvelteKit is a good fit for a given use case.

% In this section we discuss our personal experience developing with SvelteKit in an attempt to reason about SvelteKit's expected developer experience. As we noted in \Cref{sec:methodology}, providing a heuristic analysis of SvelteKit's DX is out of scope.

% In our experience, all discussed SvelteKit implementations enabled an efficient and smooth development experience. Especially the full stack approach significantly reduced overhead, because models could be reused between client and server and SvelteKit functionality handled communication.

\subsection{Project Setup}
SvelteKit provides a command line utility to create new projects (\Cref{fig:project-setup}). The utility can be run with npm and provides a guided Wizard to select configuration options. This utility provides options to configure TypeScript for type checking, ESLint\footnote{\url{https://eslint.org/}} for code linting, Prettier\footnote{\url{https://prettier.io/}} for code formatting, as well as Vitest\footnote{\url{https://vitest.dev/}} and Playwright\footnote{\url{https://playwright.dev/}} for testing. 

% \begin{itemize}
%     \item Fast, \mintinline{text}{npm cgeate svelte@latest}
%     \item missing out-of-the-box features: form validation/parsing
% \end{itemize}

\begin{figure}
    \centering
    \includegraphics[width=.95\linewidth,trim={0 15cm 0 1.5cm},clip]{assets/sveltekit-project-setup}
    \caption{Tool for creating a new SvelteKit project}
    \label{fig:project-setup}
\end{figure}

% \subsection{Syntax}
% We experienced Svelte's syntax as easy to grasp. This is because Svelte's syntax closely adheres to HTML with some added features for templating and reactivity.

% Svelte's decision to reuse assignment operators for reactivity, not only makes for a less verbose, but also more intuitive syntax. Something, something code snippet:




% \begin{myminted}{svelte}{Svelte}
% <script>
%     let count = 0;
% </script>

% <button on:click={() => (count++)}>+</button>
% \end{myminted}

% \begin{myminted}{jsx}{React}
% import { useState } from 'react';

% export function Component() {
%     const [count, setCount] = useState(0);

%     return <button onClick={() => setCount(c => c + 1)}>+</button>
% }
% \end{myminted}

% This example clearly shows how Svelte simplifies a common use case, by utilizing its compiler based approach.

% \todo{}

% Simple Reactive Statements:
% \begin{myminted}[escapeinside=||]{svelte}{}
% <script>
%     let count = 0;
%     |\$|: doubled = count * 2;
% </script>
% \end{myminted}

% \subsection{Library Interoperability}

% \todo{}

% \begin{itemize}
%     \item Nice interop because Svelte is close to vanilla HTML/JS/CSS
%     \item Svelte Code is valid HTML/JS/CSS Syntax $\rightarrow$ Tools work out of the box (e.g Prettier, ESLint)
% \end{itemize}


\subsection{Component Libraries in Svelte}
\label{sec:evaluation-ui-libs}
As described in \Cref{sec:implementation-ui}, the implementation was required to follow the SAP Fiori Design Guidelines. To this end we tried using both UI5 web components and a plain CSS classes approach using SAP Fundamental Styles. In either case we decided to create a library of Svelte components that wraps the actual UI utility. This decision was made to reduce the amount of boilerplate code and repetition required to use some UI elements. Furthermore, this decision made it easier to migrate from UI5 web components to fundamental styles, because only the component library had to be changed. 

Custom components can reduce the amount of code duplication by encapsulating UI elements into a reusable block. Therefore, components are an important tool in modern frontend design. Nonetheless, we noticed some shortcomings when working with custom components in Svelte. Notably, directives (\Cref{sec:svelte-directives}) do not work on custom components. One might for example expose \mintinline{text}{class} and \mintinline{text}{style}-attributes on a component to make it possible to pass extra styling to it (\Cref{fig:evaluation-svelte-component-style}).

\begin{listing}[H]
\begin{myminted}{svelte}{MyButton.svelte}
<script>
  let clazz;
  export { clazz as class };
  export let style;
</script>

<button class={clazz} {style}><slot/></button>
\end{myminted}
\begin{myminted}{svelte}{App.svelte}
<MyButton class="mt-2" style="background: red">Click me!</MyButton>
\end{myminted}
\caption{Svelte Component that provides a class and style attribute.}
\label{fig:evaluation-svelte-component-style}
\end{listing}

Because Svelte has no notion that these attributes correspond to \mintinline{text}{class} and \mintinline{text}{style}-attributes of an HTML-element, it is not possible to use directives to set these values. This means that the consumer of a component library is missing useful features provided by Svelte. Passing style classes especially caused further problems. As noted in \Cref{sec:svelte-styling}, Svelte scopes style definitions to the component they are declared in. This means that it is not possible to pass styles down to a child component without explicitly declaring a style as global.

\begin{listing}[H]
\begin{myminted}{svelte}{App.svelte}
<MyButton class="highlight">Click me!</MyButton>

<style>
  .highlight:hover {
    filter: brightness(150%);
  }
</style>
\end{myminted}
\caption{The style will not apply to the custom component because it is scoped.}
\label{fig:evaluation-styling-scope}
\end{listing}

In \Cref{fig:evaluation-styling-scope} a style is defined to highlight the button on hover. But because Svelte scopes the rule to the component it is defined in, the style will not work. Svelte provides the possibility to mark style as global, but this means that every element in the application could potentially be affected by it. This can cause unintended side effects and is therefore not recommended. Another way to handle this issue would be to use libraries that are designed around global styles. Tailwind CSS\footnote{\url{https://tailwindcss.com/}} for example defines all its styling utilities as global CSS classes and therefore is not affected by this problem (\Cref{fig:evaluation-styling-tailwind}).

\begin{listing}[H]
\begin{myminted}{svelte}{App.svelte}
  <MyButton class="hover:brightness-150">Click me!</MyButton>
\end{myminted}
\caption{A fixed version of \Cref{fig:evaluation-styling-scope} that uses Tailwind CSS.}
\label{fig:evaluation-styling-tailwind}
\end{listing}

% In our implementation we experimented with different approaches to structuring our UI-code

% \begin{itemize}
%     \item Most directives do not work on Components (use:, class:, style: )
%     \item Scoped styles means, it is annoying to pass down locally defined CSS-classes to children.
%     \item But, Tailwind CSS can make things better (because everything is defined globally anyway).
% \end{itemize}


% \subsection{Cases where Svelte is bad}
% While SvelteKit can provide an improved development experience, it does not come without its flaws. During out study we identified multiple areas where SvelteKit caused problems, or could become a paint point in larger projects.

% \begin{itemize}
%     \item Centralized load function (compared to RSC's, load where needed)
%     \item Things Which require node.js server (e.g. cron, websockets)
%     \item Big data sets, which need to be shared between routes (virtuelle-fabrik)
%     \item Style is scoped per default, thus it is annoying to pass classes to child component (improved with tailwind, because everything is global)
%     \item Differences between client renderer and server renderer
%     \item No best practices yet
%     \item Problems with Svelte's Custom fetch
% \end{itemize}

\subsection{Universal vs. Server Load Functions}
\label{sec:evaluation-universal}
As noted in \Cref{sec:implementation-redirect}, we encountered multiple problems with the frontend only implementation that utilizes universal load functions. The primary problem is that SvelteKit provides no support for sending data to the backend. While data loading is streamlined through universal load functions, server actions, as used in the full stack implementation, can only be used server side. This means that submitting data has to be handled manually. Furthermore, submitting data will only work when JavaScript is enabled. This is likely the main reason why SvelteKit provides no support for this use case, because it prioritizes progressive enhancement.

% \begin{myminted}{svelte}{}
% <script>
%   async function handleSubmit({cancel, request}) {
%     cancel();

%     const res = await fetch('api/update', {
%         method: 'POST',
%         body: request.body
%     });

%     if (!res.ok) {
%       // error handling...
%     }

%     return ({ update }) => {
%       update();
%     }
%   }
% <script>

% <form method="POST" use:enhance={handleSubmit}>
%   <button type="submit">send</button>
% </form>
% \end{myminted}

% During this approach we also realized further complications. Firstly, authentication with the backend API becomes more complex, because both client and server need the access token for the API.
% could this be fixed by passing through the api token??

Furthermore, using fetch in universal load functions requires usage of a special fetch function supplied by SvelteKit. This special fetch function makes sure that fetching data behaves the ame way on the client and server. Additionally, when a page with a universal load function is accessed, the load function is first run on the server during SSR, and the result is sent to the client. During hydration on the client side, the load function is then run again. This means, that ordinarily all fetch calls executed in the load function would run two times, once on the server and once on the client. To prevent this, SvelteKit's implementation inlines fetch responses on the server side, and then uses the inlined response on the client during hydration. This means, that it is advisable to use SvelteKit's fetch implementation. But, as this special fetch function is only exposed as an argument in load functions, one has to always pass this fetch function on any delegates, which require network communication. Furthermore, it is not possible to use custom HTTP clients, such as Axios\footnote{\url{https://axios-http.com/}}.

% Svelte's custom fetch function further increased complexity. SvelteKit's implementation of fetch makes sure that fetch behaves similarly on the client and server. Therefore, it enables relative URL paths on the server, which would ordinarily not be possible.

% - problems with universal
% - SvelteKit fetch weird
% - auth more complicated
% - 


\subsection{Centralized Load Function}
SvelteKit's architecture enforces that all asynchronous data has to be acquired inside a page's load function for it to be available during server side rendering. This provides a clean and easy to understand flow for simple implementations. But it can increase complexity in cases where a single isolated component has to be reused in multiple different routes.

One could for example imagine a simple component which displays a Twitter feed, which has to be used in multiple places. \Cref{fig:evaluation-reusable-file-hierarchy} outlines the directory structure for an example where both route a and b need to use a shared component that fetches and renders a Twitter feed. The way to implement this in SvelteKit would be to fetch the data in the load function of both routes and pass it to the twitter component as a prop (\Cref{fig:evaluation-reusable-sveltekit}).

\begin{listing}[h]
\begin{myminted}{html}{}
routes/
  a/
    +page.js
    +page.svelte
  b/
    +page.js
    +page.svelte
  TwitterComponent.svelte
\end{myminted}
\caption{Directory hierarchy that has a reusable component.}
\label{fig:evaluation-reusable-file-hierarchy}
\end{listing}


\begin{listing}[H]
\begin{myminted}{js}{a/+page.js b/+page.js}
export async function load({ fetch }) {
  const twitterFeed = (await fetch('/api/twitterfeed')).json();

  return {
    twitterFeed,
    // other page data...
  }
}

\end{myminted}
\begin{myminted}{svelte}{a/+page.svelte b/+page.svelte}
<script>
  import TwitterComponent from '../TwitterComponent.svelte';

  export let data;
</script>

<TwitterComponent feed={data.twitterFeed} />
<!-- other page markup... -->
\end{myminted}
\begin{myminted}{svelte}{TwitterComponent.svelte}
<script>
  export let feed
</script>

{#each feed as tweet}
  <!-- ... -->
{/each}
\end{myminted}
\caption{Example Implementation of a reusable component in SvelteKit}
\label{fig:evaluation-reusable-sveltekit}
\end{listing}

This pattern needs to be repeated for each route that wants to use the twitter component. This not only results in unnecessary code duplication, but can also cause problems later in a project's lifecycle. If the twitter component is to be removed later on, it is imaginable that one of the fetch calls is left in by accident, because the data fetching is detached from where the data is actually used, thus causing unnecessary traffic.

Other approaches circumvent this problem by coupling data fetching more closely to its usage. For example, a similar implementation using React could fetch the required data inside the component (\Cref{fig:evaluation-reusable-rsc}).

\begin{listing}[H]
\begin{myminted}{jsx}{TwitterComponent.jsx}
export function TwitterComponent() {
  const [twitterFeed, setTwitterFeed] = useState([]);
  useEffect(() => {
    fetch('/api/twitterfeed')
      .then(res => res.json())
      .then(data => setTwitterFeed(data))
  }, []);

  return <>
    {twitterFeed.map(tweet => (
      // ...
    ))}
  </>
}
\end{myminted}
\begin{myminted}{jsx}{a/page.jsx b/page.jsx}
import TwitterComponent from '../TwitterComponent';

export async function Page() {
  return <>
    <TwitterComponent />
    { /* other page markup... */ }
  </>
}
\end{myminted}
\caption{Example of a reusable component using React server components.}
\label{fig:evaluation-reusable-rsc}
\end{listing}

With this approach, the Twitter component becomes truly isolated from the page, because it can fetch its own data. On the other hand, because SvelteKit's approach enforces such a strict architecture, it is immediately clear where data fetching happens. This can have benefits when trying to understand existing code bases.

% We expect this architecture to scale better in large applications, than SvelteKit's centralized load function. 

\subsection{Reactivity Pitfalls}
\label{sec:evaluation-reactivity pitfalls}
While Svelte's reactivity is overall very intuitive, it nonetheless has some pitfalls.

Svelte's reactive statements (\Cref{sec:svelte-reactivity}) work by analyzing at compile time which variables are referenced inside a reactive statement. These variables are tracked as dependencies and every time one of these dependencies changes, Svelte reevaluates the reactive statement. But in cases where the dependency is used indirectly, for example through a function call, the dependency cannot be seen by Svelte's compiler. Therefore, updates to the dependency will not trigger reevaluation of the reactive-statement.

\begin{listing}[h!]
\begin{myminted}[escapeinside=||, autogobble]{svelte}{}
<script>
  let count = 1;

  |\$|: doubled = calcDoubled();

  function calcDoubled() {
    return count * 2;
  }
</script>

<button on:click={() => (count++)}>+</button>
<div>{count} * 2 = {doubled}</div>
\end{myminted}
\caption{\mintinline{text}{doubled} will not be recalculated when \mintinline{text}{count} changes.}
\label{fig:evaluation-reactivity-hidden-dependency}
\end{listing}

\Cref{fig:evaluation-reactivity-hidden-dependency} illustrates this behavior. In this example, the function \mintinline{svelte}|calcDoubled| calculates the value of \mintinline{svelte}|doubled| in correspondence to \mintinline{svelte}|count|. But, because \mintinline{svelte}|count| is not directly referenced in the reactive statement, it is not tracked as a dependency. Therefore, \mintinline{text}|doubled| will not be updated when \mintinline{text}{count} changes. To fix this example, \mintinline{text}{count} needs to be passed as an explicit parameter to this function. In this way the Svelte compiler can detect \mintinline{text}{count} as a dependency of \mintinline{text}{doubled} and will create the correct update statements (\Cref{fig:revaluation-reactivity-fixed}). \Cref{sec:evaluation-svelte-future} will discuss a future update to Svelte that would fix this problem.

\begin{listing}[H]
\begin{myminted}[escapeinside=||, autogobble]{svelte}{}
<script>
  let count = 1;

  |\$|: doubled = calcDoubled(count);

  function calcDoubled(c) {
    return c * 2;
  }
</script>
\end{myminted}
\caption{Fixed version of \Cref{fig:evaluation-reactivity-hidden-dependency}.}
\label{fig:revaluation-reactivity-fixed}
\end{listing}

\subsection{Ecosystem Maturity}
The original implementation uses Keycloak with OIDC as authorization mechanism. The SvelteKit frontend and SPA implementations had to therefore implement authentication with the Keycloak service. To log in with Keycloak, the user is first redirected to the Keycloak login page, where they have to then enter their credentials. After successful login, the user is redirected back to the application with an access token. This access token can then be used to authenticate with the Spring backend. In the SvelteKit frontend implementation the SvelteKit server is sending the actual requests to the Spring backend whereas in the SPA implementation, backend requests are sent client-side. This difference made it impossible to use the same authentication library for both implementations. 

To integrate authentication into the SPA implementation we used \mintinline{text}{oidc-client-ts}\footnote{\url{https://www.npmjs.com/package/oidc-client-ts}}, a stable and battle tested library. Thanks to Svelte's adherence to web standards, the library was easily hooked up with SvelteKit, without any major problems. On the other hand, authentication in the SvelteKit frontend implementation requires stronger integration with SvelteKit itself because the backend needs to be able to use the users access token to communicate with the Spring backend while also handling authentication between SvelteKit client and server. To this end, we used Auth.js\footnote{\url{https://authjs.dev/}} a general purpose authentication solution with support for many authentication providers. As of this writing, SvelteKit support for Auth.js is still experimental. This became apparent, because Auth.js did not implement features such as automatic access token refresh.

This highlights another important point about SvelteKit. The framework tries to adhere to web standards where possible. Because of this, it is often possible to use general purpose JavaScript libraries without any special adjustments. On the other hand, in cases where tighter integration with SvelteKit is required SvelteKit's relative novelty becomes apparent in the absence of mature and well-developed libraries for certain use cases.

\subsection{Deployment}
SvelteKit provides first-class adapters that automate creating production deployments for various hosting providers such as Vercel\footnote{\url{https://vercel.com/}}, Netlify\footnote{\url{https://www.netlify.com/}}, Cloudflare Pages\footnote{\url{https://pages.cloudflare.com/}}, and Cloudflare Workers\footnote{\url{https://workers.cloudflare.com/}}. Furthermore, adapters exist to create standalone Node.js server, or generate a static bundle that can be served from a simple file-base web server. Beyond this, a wide range of community plugins exist which further extend the range of platforms SvelteKit can be deployed to. This means that SvelteKit generally supports great flexibility for deployment with comparatively little effort.  

But this wide range of supported platforms has some shortcomings. SvelteKit does mostly not implement features that require platform specific functionality. One such feature would be a hook that is executed when the application shuts down. This can sometimes be required for example to close open connections. While this can be easily worked around in a Node.js environment, for example by using \mintinline{js}{process.on('sigint', ...)}, other features are not added this easily. 

SvelteKit wants to be a serverless framework \cite{noauthor_expose_nodate}. With many serverless platforms not having support for websockets yet \cite{noauthor_vercel_nodate}, websockets is a feature that is not yet supported in SvelteKit out of the box. While it is possible to add websocket support to the Node.js platform, this has limitations. By default, the Node.js adapter creates an entry point that starts a web server to host all handlers required by SvelteKit. It is possible to forgo this entry point and instead write a custom server that uses the handlers directly. In this way it is possible to create a web server that uses websockets. But, this approach only works for the production build. In development another workaround is required that works with hot-module replacement.

\subsection{Future of Svelte}
\label{sec:evaluation-svelte-future}
In September 2023, the Svelte team gave a first preview of the next major API planned for version 5 of Svelte \cite{team_introducing_2023}. The headline feature of this new iteration is called runes. Runes are a set of compiler instructions that replace the system of reactivity currently used in Svelte (\Cref{sec:svelte-reactivity}). With runes, Reactivity becomes more explicit. Instead of every top level variable in a svelte file being reactive by default, state that should be reactive has to be marked with the \mintinline{svelte}{$state} rune (\Cref{fig:svelte-state-rune}).

\begin{listing}[H]
\begin{myminted}[escapeinside=||]{svelte}{}
<script>
  let count = |\$|state(0);
</script>
\end{myminted}
\caption{The \mintinline{html}{$state} rune will be used in Svelte 5 to mark a variable as reactive.}
\label{fig:svelte-state-rune}
\end{listing}

While initially being more verbose, this has multiple advantages. Firstly, this syntax is more explicit, clearly marking which variable is reactive and which is not. Secondly, in the current reactivity system of Svelte only top-level variables are reactive. With runes, it will be possible to mark variables as reactive, anywhere in the code. Further, it will even be possible to place state runes in JavaScript files. This will make it easier to move out state into separate files. Whereas before, large rewrites were necessary to move state logic out of a Svelte file, with this new syntax it will be possible to use the same syntax in Svelte files and in JS files.

% \begin{myminted}{svelte}{counter.js}
% export function createCounter() {
%     let value = $state(0);

%     return {
%         get value() { return value },
%         set value(val) { value = val }
%     }
% }
% \end{myminted}
% \s{$}
% \begin{myminted}{svelte}{app.svelte}
% <script>
%     import { createCounter } from './counter';

%     const counter = createCounter();
% </script>
% <input bind:value={counter.value}>
% \end{myminted}

Furthermore, Svelte 5 also introduces the \mintinline{svelte}{$derived} and \mintinline{svelte}{$effect} runes which replace the \mintinline{svelte}{$:} syntax (\Cref{fig:evaluation-effect-rune}). In the current version of Svelte, \mintinline{svelte}{$:} is used to handle two different features, reactive binding of variables and defining side effects.   

\begin{listing}[H]
\begin{myminted}[escapeinside=||]{svelte}{}
<script>
  let count = |\$|state(0);

  const doubled = |\$|derived(count * 2);

  |\$|effect(() => {
    console.log(`new value of count: ${count}`);
  })    
</script>
\end{myminted}
\s{$}
\caption{Example usage of the \mintinline{html}{$derived} and \mintinline{html}{$effect} runes.}
\label{fig:evaluation-effect-rune}
\end{listing}

The new rune system will also solve the problems outlined in \Cref{sec:evaluation-reactivity pitfalls}. Currently, reactive dependencies are tracked at compile-time. This makes it impossible to track indirect dependencies. Runes on the other hand, will track their dependencies at run-time, therefore making it possible to also track indirect dependencies. This will make Svelte's system for reactivity overall easier to reason about. 

\subsection{Stability}
\label{sec:evaluation-stability}
Historically, the Svelte team did not shy away from innovating the framework. Both Svelte 2 and Svelte 3 introduced major breaking changes \Cite{harris_svelte_2018,harris_svelte_2019}. Furthermore, with runes, Svelte is already receiving the next major breaking change. The old system for reactivity is planned for removal with Svelte 7. In the meantime, Svelte will support both syntaxes in the same code base on a per component basis. This gives projects time to migrate to runes. Nonetheless, especially in large code bases this can cause costly migrations. For past breaking changes the Svelte team provided tools to automate much of the migration work \cite{harris_svelte_2018}. They aim to do the same for runes, hopefully minimizing the effort required to migrate. 

Version 1.0 of SvelteKit was released only recently in December 2022 \cite{team_announcing_2022}. It is not possible to say how stable SvelteKit's APIs will be going forward. If Svelte's history is any indication, SvelteKit will probably also receive major API overhauls, should the need arise. Historically, Svelte has proven to be fast moving, therefore projects that use SvelteKit should be ready to put in regular migration work, to prevent the code base from becoming outdated.

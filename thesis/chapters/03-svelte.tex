\section{Svelte}
\label{sec:svelte}

Svelte was created in 2016 by Rich Harris as a successor to Ractive.js \cite{offerzen_origins_svelte_2022}. It is a framework for developing web user interfaces, similar to React, Angular, and Vue.js. Harris describes Svelte as a framework not to run code but to think about code \cite{offerzen_origins_svelte_2022}. Svelte's key difference compared to other web UI frameworks is, that it uses a compiler. This makes it possible to shift many rendering steps from the run time into the build time. Therefore, Svelte can ship smaller and more efficient bundles. 

Svelte uses its compiler to overload existing JavaScript syntax with special meaning. This allows for language extensions while still allowing it to be processed by existing JavaScript tooling (e.g. syntax highlighters).

\subsection{Features}

In the following we provide an overview of how Svelte's syntax is structured and showcase some of Svelte's features.

\subsubsection{Svelte Files}
\label{sec:svelte-file}
% - Files with .svelte ending
% - Syntax very similar to regular html with some extensions
% - standard javascript/Typescript $=>$ existing tools can be more easily integrated
% - javascript/html/css colocated in same file
% - {} to use js expressions in html, similar to react.
% - shorthand attribute syntax

Svelte is a custom language that is very similar to HTML, JavaScript, and CSS, which is colocated inside a \mintinline{text}{.svelte} file. In fact, plain HTML syntax is also valid Svelte syntax (\Cref{fig:svelte-basic-html}).

\begin{listing}[h!]
\begin{myminted}{svelte}{app.svelte}
<h1>Hello World!</h1>
\end{myminted}
\caption{HTML is valid Svelte-syntax.}
\label{fig:svelte-basic-html}
\end{listing}

Similarly to HTML, JavaScript can be added inside \mintinline{text}{<script>} tags. But Svelte further extends this syntax to enable closer integration between JavaScript and HTML. Outside the script-block it is possible to use curly braces to write arbitrary JavaScript expressions. These expressions can reference variables and functions defined inside the script block (\Cref{fig:svelte-javascript-expression}).

% Logic can be added with a \mintinline{svelte}{<script>}-block and variables can be used outside the HTML with curly-braces:

\begin{listing}[h!]
\begin{myminted}{svelte}{}
<script>
  const name = 'world';
</script>

<div>Hello {name.toUpperCase()}!</div>
\end{myminted}
\caption{Example for JavaScript expressions in Svelte.}
\label{fig:svelte-javascript-expression}
\end{listing}

Similar to React, variables can also be used in element attributes, and Svelte even provides a shorthand when variable name and attribute name are the same (\Cref{fig:svelte-attribute-variables}).

\begin{listing}[h!]
\begin{myminted}{svelte}{}
<script>
  const src = './some/img.svg';
</script>

<img src={src}>
<img {src}> <!-- equivalent to <img src={src}> -->
\end{myminted}
\caption{Syntax for using variables in element attributes.}
\label{fig:svelte-attribute-variables}
\end{listing}

\subsubsection{Reactivity}
\label{sec:svelte-reactivity}
% - asignments with usual js asignments. would not work in React.
% - unlike react, svelte runs script just once, for assignments, svelte inserts invalidate blocks to update a variable
% - compare svelte to equivalent react code.

Reactive programming is a programming paradigm that enables development of declarative event-driven applications \cite{bainomugisha_survey_2013}. In reactive programming the developer only declares what to do, whereas when to do it is managed by the language. To this end, the language has to track dependencies and automatically propagate changes across these dependencies when they happen.

Svelte provides a strong system for reactive programming. Variable changes are automatically tracked and trigger UI updates. Therefore, it is possible to use regular assignment operations to mutate variables, which is not possible in frameworks such as React (\Cref{fig:svelte-basic-reactivity}).

\begin{listing}[H]
\begin{myminted}{svelte}{}
<script>
  let count = 0;

  function increment() {
    count += 1;
  }
</script>

<div>count: {count}!</div>
<button on:click={increment}>click me!</button>
\end{myminted}
\caption{Basic example for mutation of variables in Svelte.}
\label{fig:svelte-basic-reactivity}
\end{listing}

To achieve this, Svelte's compiler replaces assignment operations with operations that automatically trigger changes of the reactive dependency chain at compile time. The increment function in the prior example would be compiled to an invalidation statement that tells the UI to update (\Cref{fig:svelte-compiler-output}).

\begin{listing}[H]
\begin{myminted}[escapeinside=||]{js}{}
function increment() {
  |\$\$|invalidate('count', count += 1)
}
\end{myminted}
\caption{Compiler output for the \mintinline{text}{increment} function defined in \Cref{fig:svelte-basic-reactivity}}
\label{fig:svelte-compiler-output}
\end{listing}
% An equivalent React implementation using function components would have to use more verbose semantics, because without a specialized compiler, assignment operations do not have any special meaning:

% \begin{myminted}{jsx}{}
% function Component() {
%   const [count, setCount] = useState(0);

%   function increment() {
%     setCount(c => c + 1);
%   }

%   return <>
%     <div>count: {count}!</div>
%     <button onClick={increment}>click me!</button>
%   </>
% }
% \end{myminted}

% React's code runs every time, a component is updated, this means derived values can just be declared, and they are evaluated every time. Svelte's code on the other hand, only runs once on component initialization. Therefore, Svelte introduces custom syntax for defining reactive statements. To This end, Svelte repurposes labeled statements:

Beyond reactive \textit{UI updates} following mutation of a variable, Svelte also provides syntax to trigger \textit{computations} following mutation of variables (\Cref{fig:svelte-reactive-variable-syntax}).

\begin{listing}[H]
\begin{myminted}[highlightlines={4},escapeinside=||]{svelte}{}
<script>
  let count = 2;

  |\$|: doubled = count * 2;

  function increment() {
    count += 1;
  }
</script>

<button on:click={increment}>click me!</button>
<div>count: {count}!</div>
<div>doubled: {doubled}!</div>
\end{myminted}
\caption{Example of a reactive variable in Svelte. \mintinline{text}{doubled} will automatically be recalculated when \mintinline{text}{count} changes.}
\label{fig:svelte-reactive-variable-syntax}
\end{listing}
% - $: to make reactive statement
% - repurpose of rarely used js feature
% - used for reactive assignments, conditions, and whole codeblocks
% - reactive block is rerun everytime a variable used inside the block changes

\mintinline{js}{$:} marks a statement as reactive. Every time a value, that is used in a reactive statement, changes, the statement is reevaluated. This can be used to declare reactive variables, as seen in the prior example, but it can also be used to run arbitrary code blocks (\Cref{fig:svelte-reactive-block-syntax}).

\begin{listing}[H]
\begin{myminted}{svelte}{}
<script>
  let count = 0;

  $: {
    console.log(`new value of count: ${count}`);
    console.log('This will also be executed when count changes');
  }

  $: if(count > 9) {
    console.log('It is over 9!');
    count = 0;
  }
</script>
\end{myminted}
\s{$}
\caption{Svelte example that runs statements when \mintinline{text}{count} changes}
\label{fig:svelte-reactive-block-syntax}
\end{listing}

\subsubsection{Templates}

% - Templates for if, each

HTML does not have a way of expressing logic, thus Svelte introduces its own syntax. To conditionally render some markup, the markup has to be wrapped inside an if block (\Cref{fig:svelte-if-syntax}).

\begin{listing}[H]
\begin{myminted}{svelte}{}
<script>
  let count = 0;
</script>

{#if count > 9}
  <div>It's over 9!!</div>
{:else}
  <div>The number is pretty low</div>
{/if}
\end{myminted}
\caption{Example usage of Svelte's if block.}
\label{fig:svelte-if-syntax} 
\end{listing}

UIs also often work with lists of data. To this end, Svelte provides an each block, which can be used to handle iterable values such as arrays (\Cref{fig:svelte-array-syntax}).

\begin{listing}[H]
\begin{myminted}{svelte}{}
<script>
  let array = [1, 2, 3, 4];
</script>

{#each array as value}
  <div>{value}</div>
{/each}
\end{myminted}
\caption{Example usage of Svelte's each block.}
\label{fig:svelte-array-syntax} 
\end{listing}

\subsubsection{Components}

% - Since emergence of React, component driven development became norm.
% - in Svelte, one file = one component.
% - Name = Component Name.

Since the emergence of React in 2013 \cite{occhino_js_2013}, component-driven user interfaces have become a widespread approach to web development. In Svelte, a component is represented by a singular \mintinline{html}{.svelte} file, where the file name determines the component name (\Cref{fig:svelte-component-basic}).

\begin{listing}[H]
\begin{myminted}{html}{MyComponent.svelte}
<script>
  export let value;
</script>

<div>hello, {value}!</div>
\end{myminted}
\caption{Basic Svelte component that defines the input \mintinline{text}{value}.}
\label{fig:svelte-component-basic}
\end{listing}

To define inputs that should be passed into the Component, Svelte repurposes the \mintinline{text}{export} keyword. In the prior example \mintinline{svelte}{value} is defined as an input of the Component \mintinline{text}{MyComponent}. The Component can than be imported in another Svelte file and be used like a regular element (\Cref{fig:svelte-component-usage}).

\begin{listing}[H]
\begin{myminted}{svelte}{app.svelte}
<script>
  import MyComponent from './MyComponent.svelte';

  let value = 'world';
</script>

<MyComponent value={value} />
\end{myminted}
\caption{Example usage of a custom Svelte component.}
\label{fig:svelte-component-usage}
\end{listing}

\subsubsection{Events}

In a prior example (\Cref{fig:svelte-reactive-variable-syntax}) we already showed usage of \mintinline{svelte}{on:click}. This directive is used to listen to mouse clicks. Svelte makes it possible to listen for arbitrary events on an element with the \mintinline{html}{on} keyword (\Cref{fig:svelte-on-directive}). Furthermore, Svelte provides an event dispatcher to enable components to send their own custom events (\Cref{fig:svelte-custom-event}).

\begin{listing}[H]
\begin{myminted}[highlightlines={7}]{html}{}
<script>
  function handleMove(event) {
    // ...
  }
</script>

<div on:pointermove="{handleMove}" ></div>
\end{myminted}
\caption{Usage of the \mintinline{text}{on:} directive to listen to \mintinline{text}{pointermove} events.}
\label{fig:svelte-on-directive}
\end{listing}


\begin{listing}[H]
\begin{myminted}{html}{MyComponent.svelte}
<script>
  import { createEventDispatcher } from 'svelte';

  const dispatch = createEventDispatcher();

  function sayHello() {
    dispatch('message', { text: 'Hello!' });
  }
</script>
\end{myminted}
\begin{myminted}{svelte}{app.svelte}
<script>
  import MyComponent from './MyComponent.svelte';
</script>

<MyComponent on:message={(e) => console.log(e.detail.text)} />
\end{myminted}
\caption{Usage of the event dispatcher to send custom events.}
\label{fig:svelte-custom-event}
\end{listing}

\subsubsection{Styling}
\label{sec:svelte-styling}
Like most web frameworks, Svelte uses CSS to style content. To this end, it is possible to use \mintinline{text}{<style>} tags to define CSS similar to HTML (\Cref{fig:svelte-style-tag}). Svelte automatically scopes defined styles. This means, the style will only affect markup that is in the same file as the style. This can prevent unexpected styling behavior in growing code bases. 

\begin{listing}[H]
\begin{myminted}{svelte}{}
<button>Click me!</button>

<style>
  button:hover {
    background: red;
  }
</style>
\end{myminted}
\caption{Example usage of \mintinline{text}{<style>} tag to add CSS.}
\label{fig:svelte-style-tag}
\end{listing}

\subsubsection{Directives}
\label{sec:svelte-directives}
In the previous section we introduced the \mintinline{text}{on:} keyword to react to events. This keyword is called a directive. Svelte provides multiple directives to control the behavior of components in some ways.

The \mintinline{svelte}{bind:} directive can be used to create two-way bindings between parent and child elements. In Svelte, data usually flows from parent to child with the child having no way of directly mutating the data that it is being passed. By passing a property using the \mintinline{svelte}{bind:} directive, Svelte enables data to flow the other way (\Cref{fig:svelte-bind-directive}).

\begin{listing}[H]
\begin{myminted}{svelte}{}
<script>
  let name = 'Peter';
</script>

<input bind:value={name} />
\end{myminted}
\caption{Two-way binding with the \mintinline{text}{bind:}-directive.}
\label{fig:svelte-bind-directive}
\end{listing}

The \mintinline{svelte}{class:} directive is a utility that can be used to dynamically set CSS classes on an element. This directive takes an expression that when evaluation to true, will set the given class on the element. This directive also provides a shorthand in cases where a variable is used to set a class of the same name (\Cref{fig:svelte-class-directive}).
\begin{listing}[H]
\begin{myminted}{svelte}{}
<script>
  let active = true;
</script>

<!-- all tree declarations are equivalent -->
<div class={active ? 'active' : ''} />
<div class:active={active} />
<div class:active />
\end{myminted}
\caption{Example usage of the \mintinline{text}{class:} directive.}
\label{fig:svelte-class-directive}
\end{listing}

Similar to the previous directive, the \mintinline{svelte}{style:} directive provides a utility for dynamically setting CSS-styles on an element. The directive takes a value that is used to set the style attribute, where \mintinline{text}{undefined} and \mintinline{text}{null} means that the attribute should not be set. In cases where a variable is used to set a style attribute of the same name, Svelte also provides a shorthand (\Cref{fig:svelte-style-directive}).

\begin{listing}[H]
\begin{myminted}{svelte}{}
<script>
  let color = "red";
  let width = "20rem";
</script>

<!-- all tree declarations are equivalent -->
<div style="color:red; width:20rem" />
<div style:color="red" style:width="20rem" />
<div style:color style:width />
\end{myminted}
\caption{Example usage of the \mintinline{text}{style:} directive.}
\label{fig:svelte-style-directive}
\end{listing}

The \mintinline{text}{use:} directive is used to register an action on an HTML element. An action is a function that is called when the element is created. The function receives a reference to the element and can thus be used to augment behavior of an element. In \Cref{fig:svelte-use-directive} the \mintinline{text}{use:} directive is used to register event listeners that log to the console when a mouse enters the element. While this functionality can be easily implemented using the \mintinline{text}{on:} directive, actions become especially useful when more complex behavior has to be configured in a reusable way. \Cref{sec:sveltekit-form-actions} will introduce an action that progressively enhances form elements.

\begin{listing}[H]
\begin{myminted}{svelte}{}
<script>
  function listen(node) {
    node.addEventListener('mouseenter', () => console.log("mouse enter"))
    node.addEventListener('mouseleave', () => console.log("mouse leave"))
  }
</script>

<button use:listen>Click me!</button>
\end{myminted}
\caption{Example usage of actions to register an event listener.}
\label{fig:svelte-use-directive}
\end{listing}


\subsubsection{TypeScript}
TypeScript is a programming language that has gained a lot of popularity in recent years \cite{noauthor_stack_nodate}. It is a superset of JavaScript that adds typing and type annotations to the language. Svelte has support for TypeScript out of the box. TypeScript can be enabled for a Svelte file by annotating the script block with a \mintinline{text}{lang} attribute (\Cref{fig:svelte-ts-usage}). 
\begin{listing}[H]
\begin{myminted}{svelte}{}
<script lang="ts">
  let value: string = "World"

  function sum(a: number, b: number): number {
    return a + b;
  }
</script>
\end{myminted}
\caption{Svelte file that uses TypeScript.}
\label{fig:svelte-ts-usage}
\end{listing}

It should be noted that using TypeScript in conjunction with Svelte is currently subject to a technical limitation: it is not possible to use TypeScript syntax outside the \mintinline{text}{<script>} tag \cite{noauthor_proposal_nodate}. One might for example want to type the parameters of an inline event handler (\Cref{fig:svelte-ts-inline-error}). Fortunately, in most cases this can be fixed by refactoring the inline function to a function definition in the \mintinline{text}{<script>} tag.
\begin{listing}[H]
\begin{myminted}{svelte}{}
<input on:change={(e: InputEvent) => {}}> /* Syntax Error */
\end{myminted}
\caption{TypeScript syntax in inline functions is currently not possible.}
\label{fig:svelte-ts-inline-error}
\end{listing}

\section{SvelteKit}
\label{sec:sveltekit}

SvelteKit is a framework that builds on top of Svelte to provide a toolkit for rapidly developing web applications \cite{noauthor_sveltekit_nodate}. It can be used out of the box to develop full stack applications, server rendered frontend applications, static web pages, and traditional single page applications without any server rendering. It provides multiple features, required for the development of modern web applications. Features include routing, support for different rendering patterns, build optimization, data preloading, and zero-configuration deployments. Furthermore, SvelteKit uses Vite\footnote{https://vitejs.dev/} to enable a good development experience, with fast startup times and quick hot module replacement.

% Meta Framework
% Stable release 2022
% provides functionality for modern applications
% Deployable as static files, web server, edge functions

\subsection{Features}
In the following we give an overview of multiple SvelteKit concepts and features, which will be relevant for the rest of this study.


\subsubsection{Routing}
\label{sec:sveltekit-routing}

SvelteKit provides built-in support for routing. To this end, SvelteKit utilizes a file system based routing system. This means that the URLs that a user can access on the web application directly maps to the directory structure in the project's file system. A SvelteKit project contains a \mintinline{text}|routes| directory. This directory marks the root of the URL, e.g.: \mintinline{text}|http://example.com/|. Respectively, the directory structure \mintinline{text}|routes/about| would map to \mintinline{text}|http://example.com/about|. This system can also be used to create parameterized routes using characters that one would not normally expect within a file path. The directory structure \mintinline{text}|routes/blog/[slug]| would create a route, where \mintinline{text}|slug| serves as a wildcard (e.g.: \mintinline{text}|/blog/hello-world|, \mintinline{text}{/blog/123}). The path component matched by this wildcard can then be used in the load function (see \Cref{sec:sveltekit-loading}). The actual content of a page is defined in a \mintinline{text}{+page.svelte} file (\Cref{fig:sveltekit-page-file}).

\begin{listing}[h!]
\begin{myminted}{svelte}{routes/+page.svelte}
<h1>My Website</h1>
<p>Welcome to my website...</p>
\end{myminted}
\begin{myminted}{svelte}{routes/about/+page.svelte}
<h1>About</h1>
<p>about this site...</p>
<a href="/home">Home</a>
\end{myminted}
\caption{Example page definitions in SvelteKit.}
\label{fig:sveltekit-page-file}
\end{listing}

SvelteKit uses the HTML \mintinline{text}|<a>| tag to navigate between pages, rather than a framework specific \mintinline{text}{<Link>} component. This has the advantage of being progressively enhanceable. If JavaScript is not available on the client-side, the anchor tag's navigation event can be handled by the browser and the browser will do a regular full page reload to reach the navigation target. In the case of JS being available, SvelteKit's runtime router will intercept the navigation event and perform an in-app navigation to the target without causing a full page reload.

Most websites have some elements that should be displayed across multiple pages, such as a navigation bar or a footer. To this end, SvelteKit's router has the concept of layouts. A layout is placed alongside the pages inside the \mintinline{text}{routes} directory, inside a \mintinline{text}{+layout.svelte} file. This layout will automatically be applied to all route and all sub-routes of the directory, where the layout file is placed. The place where the actual page content should be rendered inside the layout is defined by a \mintinline{text}{<slot>} element (\Cref{fig:sveltekit-layout}).

\begin{listing}[h!]
\begin{myminted}{svelte}{routes/+layout.svelte}
<nav>
  <a href="/home">home</a>
  <a href="/blog">blog</a>
  <a href="/about">about</a>
</nav>

<slot/>
\end{myminted}
\caption{Example for a layout that provides a navigation bar.}
\label{fig:sveltekit-layout}
\end{listing}

In some cases an application may have multiple routes, that should live on the same route as others, but should have their own shared layout. One could for example imagine an application where the user authentication pages should have a different layout than the authenticated area of the app. To support this, SvelteKit provides a mechanism to group routes together without this route being reflected in the URL by wrapping the group name in parentheses (\Cref{fig:sveltekit-layout-groups}).

\begin{listing}[h!]
\begin{myminted}[highlightlines={2,6}]{bash}{}
routes/
  (restricted)/
    dashboard/
    details/
    +layout.svelte
  (public)/
    signin/
    register/
    +layout.svelte
\end{myminted}
\caption{Usage of layout groups to provide different layouts for public and restricted routes of an application.}
\label{fig:sveltekit-layout-groups}
\end{listing}

In this example two new layout groups are introduced (\mintinline{text}|(restricted)|,\mintinline{text}|(public)|). Both groups can provide their own layout without having an influence on the URL. For example, the sign-in page is available on \mintinline{text}{/signin} (not \mintinline{text}{/(public)/signin}).

\subsubsection{Loading Data}
\label{sec:sveltekit-loading}

SvelteKit facilitates a standard way for fetching dynamic content that is to be rendered on a page. This is required to enable server side rendering and static site generation. To this end, every page can have a dedicated load function that is defined in a \mintinline{text}{+page.js} file. When a user requests a page, this load function is executed and the returned data is passed to the \mintinline{text}{+page.svelte} file as special property \mintinline{text}{data} (\Cref{fig:sveltekit-loading-data}).

\begin{listing}[h!]
\begin{myminted}{js}{routes/+page.js}
export async function load() {
  const todos = (await fetch('/api/todos')).json();

  return { todos };
}
\end{myminted}
\begin{myminted}{svelte}{routes/+page.svelte}
<script>
  export let data;
</script>

{#each data.todos as todo}
  <div>{todo.name}</div>
{/each}
\end{myminted}
\caption{Example page that fetches to-dos from an API and shows them.}
\label{fig:sveltekit-loading-data}
\end{listing}

In SSR mode, when a user first requests a page, the load function is executed on the server. Afterwards the Svelte-file is rendered once, before being sent to the user as populated HTML. In the browser, SvelteKit then hydrates the page. During hydration, SvelteKit runs the load function a second time on the client to create a consistent data model.

After the app has finished hydrating, all following navigation events will be handled by SvelteKit's runtime router. When the user now navigates to a different page the load function now is invoked on the client-side directly, therefore saving an unnecessary roundtrip to the server. Because these load functions run on both server and the client, they are called universal load functions.

Some data fetching can only take place on the server side, for example database access. Therefore, SvelteKit also provides an alternative form of load function that is only executed on the server. These server load function are placed in \mintinline{text}{+page.server.js} files (\Cref{fig:sveltekit-server-load-function}).

\begin{listing}[h!]
\begin{myminted}{js}{routes/+page.server.js}
import db from '$lib/server/db';

export async function load() {
  const todos = await db.getTodos();

  return { todos };
}
\end{myminted}
\s{$}
\caption{Usage of server load functions to fetch to-dos from a database.}
\label{fig:sveltekit-server-load-function}
\end{listing}

In the same way as universal load functions, server load functions are executed on the server when a user initially requests a page. But, because they can only run on the server, when a user navigates inside the app, instead of running the load function client-side, the client sends a request to the server for new data. The server then executes the load function and sends the resulting data back to the frontend as JSON.

This behavior also highlights another difference between universal and server load functions. Because server load functions need to be able to send JSON to the client, all data they return must be serializable. This restriction does not apply to universal load functions. Therefore, they can return arbitrary data such as class instances (\Cref{fig:sveltekit-universal-load}).

\begin{listing}[h!]
\begin{myminted}{js}{routes/+page.js}
class Todo {
  constructor(name) {
    this.name = name;
  }
}

export async function load() {
  const todos =  [ new Todo(`finish master's thesis`) ]

  return { todos };
}
\end{myminted}
\caption{Universal load function that returns non-serializable data.}
\label{fig:sveltekit-universal-load}
\end{listing}

\subsubsection{Form Actions}
\label{sec:sveltekit-form-actions}

The previous section discussed how SvelteKit applications send dynamic data to the client. But, many applications also need a way to send data to the server. To this end, SvelteKit provides form actions. \mintinline{text}{+page.server.js} files can export actions which make it possible to post data to the server using HTML's \mintinline{text}{<form>} elements (\Cref{fig:sveltekit-server-action}).

\begin{listing}[h!]
\begin{myminted}{svelte}{routes/+page.server.js}
export const actions = {
  postTodo: async () => {
    // post Todo
  }
}
\end{myminted}
\begin{myminted}{svelte}{routes/+page.svelte}
<form method="POST" action="?/postTodo">
  <label>
    Todo name:
    <input name="todoName">
  </label>
</form>
\end{myminted}
\caption{Example server action to post a new to-do.}
\label{fig:sveltekit-server-action}
\end{listing}

The name of the action, the form should post to, is specified as a query parameter. This makes it possible to have multiple actions per route. Because the implementation uses only default HTML forms, it is again possible to progressively enhance this feature. In fact, in the shown implementation, submitting the form would cause a full page reload because the browser is sending the post request to the server. To progressively enhance this form, one can add the \mintinline{text}{enhance} action (\Cref{fig:sveltekit-form-action-enhance}):

\begin{listing}[h!]
\begin{myminted}{svelte}{routes/+page.svelte}
<script>
  import { enhance } from '$app/forms';
</script>

<form method="POST" action="?/postTodo" use:enhance>
\end{myminted}
\s{$}
\caption{Progressively enhanced form that will handle submits without a full page reload.}
\label{fig:sveltekit-form-action-enhance}
\end{listing}

This action intercepts form submit events and instead posts them to the server as an asynchronous request. In case the server does not return an error, the client will then trigger the load function again to synchronize any data changes that could have happened as a result of the form action. This way the UI can forego a full page reload and therefore provide a better user experience.

\subsubsection{Rendering}

SvelteKit provides the functionality to define the rendering strategy for each route independently. By exporting the constants \mintinline{text}|ssr|, \mintinline{text}|csr| and \mintinline{text}{prerender} in a page or a layout, the rendering behavior can be precisely changed (\Cref{fig:sveltekit-rendering-config}).

\begin{listing}[h!]
\begin{myminted}{js}{routes/+page.js}
// default settings:
export const prerender = false;
export const ssr = true;
export const csr = true;
\end{myminted}
\caption{Variables that can be exported to configure the rendering behavior of a page in SvelteKit.}
\label{fig:sveltekit-rendering-config}
\end{listing}

\begin{itemize}
    \item \mintinline{text}{prerender} determines if the page should be rendered during the build process. If this set to true, SvelteKit will serve the page as static files. While this will improve performance it also means that the page cannot have form actions because SvelteKit will not provide handlers for a pre-rendered route. Furthermore, the pages load function is ran during build-time. Therefore, any external dependencies (e.g.: REST API) need to be available at this point in time.
    
    \item \mintinline{text}{ssr} determines if the page should be rendered on the server site when the user first requests the page. This does not mean the page cannot specify a server load function. Instead, the client will simply request the data from the server load function as soon as the runtime is initialized on the client side. This option can sometimes be useful, when a page requires certain features that require access to browser exclusive API's to render its content. But, most of the time it is possible to defer access to these browser-specific API's by running the code after the function has been mounted.

    \item \mintinline{text}{csr} configures if SvelteKit should run any JavaScript on the client side for this page. This way applications that do not require JavaScript in the browser, can completely disable it. But this has certain side effects. For example, given an application has a layout that contains a navigation bar which needs JavaScript to work. If the user first loads the website on a page that has CSR disabled, the navigation bar will not work. On the other hand, if the user loads the website on a page that has CSR enabled and then navigates to a page with CSR set to false, the navigation bar will work.
\end{itemize}
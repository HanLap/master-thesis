\NeedsTeXFormat{LaTeX2e}
% LTeX: enabled=false
\documentclass[a4paper,
fontsize=12pt,
headsepline,           % Linie zw. Kopfzeile und Text
oneside,               % einseitig
number=noenddot,       % keine Punkte nach den letzten Ziffern in Überschriften
bibliography=totoc,    % LV im IV
%DIV=15,               % Satzspiegel auf 15er Raster, schmalere Ränder   
BCOR=15mm              % Bindekorrektur
%,draft
]{scrbook}
\KOMAoptions{DIV=last} % Neuberechnung Satzspiegel nach Laden von Paket helvet
% \usepackage{scrhack}

\pagestyle{headings}
\usepackage{blindtext}

% für Texte in deutscher Sprache
\usepackage[ngerman]{babel}
\usepackage[utf8]{inputenc}
\usepackage[T1]{fontenc}

% Helvetica als Standard-Dokumentschrift
\usepackage[scaled]{helvet}
\renewcommand{\familydefault}{\sfdefault} 

\usepackage{graphicx}

% be able to include real vector graphics in thesis. You need inkscape installed and your pdflatex command must be called with --shell-escape for this to work.
% TeXstudio: Options → Configure → Commands → pdflatex = pdflatex -synctex=1 -interaction=nonstopmode --shell-escape %.tex
% inkscape=newer to avoid re-export if nothing has changed in the svg. inkscapelatex=false for not using latex to render text (which in turn messes up all text in image). inkscapearea = page so that the SVG size is respected and borders in the SVG are not cut off
\usepackage[inkscape=newer, inkscapelatex=false, inkscapearea=page, inkscapepath=out/_svg]{svg}

% Literaturverzeichnis mit BibLaTeX
\usepackage[babel,german=quotes]{csquotes}
\usepackage[backend=biber]{biblatex}
\addbibresource{MA.bib}
\bibliography{bibliography}

% Alternative mit Paket-Option backend=biber und \addbibresource
% \usepackage[backend=biber]{biblatex}
% \addbibresource{bibliography.bib}

% Für Tabellen mit fester Gesamtbreite und variabler Spaltenbreite
\usepackage{tabularx} 

% Besondere Schriftauszeichnungen
\usepackage{url}              % \url{http://...} in Schreibmaschinenschrift
\usepackage{color}            % zum Setzen farbigen Textes

\usepackage{amssymb, amsmath} % Pakete für Mathe-Umgebungen und -Symbole

\usepackage{setspace}         % Paket für div. Abstände, z.B. ZA
%\onehalfspacing              % nur dann, wenn gefordert; ist sehr groß!!
\setlength{\parindent}{0pt}   % kein linker Einzug der ersten Absatzzeile
\setlength{\parskip}{1.4ex plus 0.35ex minus 0.3ex} % Absatzabstand, leicht variabel

% Tiefe, bis zu der Überschriften in das Inhaltsverzeichnis kommen
\setcounter{tocdepth}{3}      % ist Standard

\usepackage{xcolor} % to access the named colour LightGray
\definecolor{LightGray}{gray}{0.9}

\usepackage[
  outputdir=out,
  ]{minted}
\setminted{
framesep=2mm,
baselinestretch=1.0,
bgcolor=LightGray,
fontsize=\footnotesize
% ,linenos
}
\input{mintedfix}
\renewcommand{\MintedPygmentize}{./pygmentize.py}


% hier Namen etc. einsetzen
\newcommand{\fullname}{Hannah Lappe}
\newcommand{\email}{hannah.lappe@uni-ulm.de}
\newcommand{\titel}{Case-study for SvelteKit in a Modern Business Application}
\newcommand{\jahr}{2023}
\newcommand{\matnr}{922114}
\newcommand{\gutachterA}{Prof.\,Dr.\,Manfred Reichert}
\newcommand{\gutachterB}{Prof.\,Dr.\,Rüdiger Pryss}
\newcommand{\betreuer}{Dr.\,Marc Schickler\\Jens Scheible}

% hier die Fakultät auswählen
%\newcommand{\fakultaet}{---  Im Quellcode anpassen nicht vergessen! ---}
\newcommand{\fakultaet}{Ingenieurwissenschaften, Informatik und\\Psychologie}
%\newcommand{\fakultaet}{Mathematik und\\Wirtschafts-\\wissenschaften}
%\newcommand{\fakultaet}{Medizin}
%\newcommand{\fakultaet}{Naturwissenschaften}

% hier das Institut einsetzen
\newcommand{\institut}{Institut für Datenbanken und Informationssysteme}

% Informationen, die LaTeX in die PDF-Datei schreibt
\pdfinfo{
  /Author (\fullname)
  /Title (\titel)
  /Producer     (pdfeTex 3.14159-1.30.6-2.2)
  /Keywords ()
}

\usepackage{hyperref}
\hypersetup{
pdftitle=\titel,
pdfauthor=\fullname,
pdfsubject={Masterarbeit},
pdfproducer={pdfeTex 3.14159-1.30.6-2.2},
colorlinks=false,
pdfborder=0 0 0	% keine Box um die Links!
}

% Trennungsregeln
\hyphenation{Sil-ben-trenn-ung}

% LTeX: enabled=true

\begin{document}
\frontmatter

% Titelseite
\thispagestyle{empty}
\begin{addmargin*}[4mm]{-10mm}

  \hfill
  \includegraphics[height=1.8cm]{assets/logo_uulm_sw.png}\\[1em]

  % LTeX: language=de-DE
  {\footnotesize
  %{\bfseries Universität Ulm} \textbar ~89069 Ulm \textbar ~Germany
  \hspace*{115mm}\parbox[t]{35mm}{\bfseries Fakultät für\\
    \fakultaet\\
    % TODO hier Institut anpassen
    \mdseries \institut}\\[2cm]

  \parbox{140mm}{\bfseries \LARGE \titel}\\[2.5em]
  {\footnotesize Abschlussarbeit an der Universität Ulm}\\[3em]

  {\footnotesize \bfseries Vorgelegt von:}\\
  {\footnotesize \fullname\\ \email}\\ \matnr\\[2em]
  {\footnotesize \bfseries Gutachter:}\\
  {\footnotesize \gutachterA\\ \gutachterB}\\[2em]
  {\footnotesize \bfseries Betreuer:}\\
  {\footnotesize \betreuer}\\\\
  {\footnotesize \jahr}
  }
\end{addmargin*}


% Impressum
\clearpage
\thispagestyle{empty}
{ \small
  \flushleft
  Fassung \today \\\vfill
  \copyright~\jahr~\fullname\\[0.5em]
  % Wenn Sie Ihre Arbeit unter einer freien Lizenz bereitstellen möchten, können Sie die nächste Zeile in Ihren Code aufnehmen. Bitte beachten Sie, dass Sie hierfür an allen Inhalten, inklusive enthaltener Abbildungen, die notwendigen Rechte benötigen! Beim Veröffentlichungsexemplar Ihrer Dissertation achten Sie bitte darauf, dass der Lizenztext nicht den Angaben in den Metadaten der genutzten Publikationsplattform widerspricht. Nähere Information zu den Creative Commons Lizenzen erhalten Sie hier: https://creativecommons.org/licenses/
  %This work is licensed under the Creative Commons Attribution 4.0 International (CC BY 4.0) License. To view a copy of this license, visit \href{https://creativecommons.org/licenses/by/4.0/}{https://creativecommons.org/licenses/by/4.0/} or send a letter to Creative Commons, 543 Howard Street, 5th Floor, San Francisco, California, 94105, USA. \\
  Satz: PDF-\LaTeXe
}

% ab hier Zeilenabstand etwas größer 
\setstretch{1.2}

\tableofcontents

\mainmatter
\chapter{Introduction}
\label{ch:introduction}
Over the past decade, single page applications (SPAs) have become the most popular approach for developing websites. This development was fueled by JavaScript frameworks such as React, Angular and Vue.js which made it easy to develop feature rich web applications. But modern web applications need to satisfy a growing number of technical requirements. Features, such as server side rendering (SSR), code splitting, application routing, and state management have become ubiquitous. This abundance of requirements has led to a point where it has become difficult to start new projects from scratch.

To address this issue, in recent years the JavaScript ecosystem saw the emergence of so-called meta frameworks. Meta frameworks are JavaScript libraries that try to fit the role of opinionated "batteries included" frameworks, known from other programming languages, such as Ruby on Rails or Django. They provide out of the box support for many of the features required for building modern web applications. They promise that developers need to spend less time thinking about software architecture and can spend more time thinking about business requirements. Furthermore, meta frameworks can use JavaScript for both frontend and backend, which allows for a tighter integration between the two. Therefore, they can enable a more streamlined development experience, as well as reducing overall application complexity.

SvelteKit is an up-and-coming meta framework that is rapidly gaining popularity. It has multiple advantages over traditional SPA frameworks. SvelteKit allows per page control over the rendering strategy. This allows to render different parts of an application with SSR, client side rendering (CSR) or at build time with pre-rendering. Furthermore, SvelteKit uses various compile time optimizations to produce a more efficient and performant JavaScript bundle. Moreover, it provides functionality, such as reactive state management, component composition, and streamlined data fetching, which help in writing concise and readable code. Overall, SvelteKit promises to make development of modern web applications faster, while simultaneously improving developer experience.

\section{Problem Statement}
\label{sec:problem-statement}
SvelteKit claims to speed up the development process by handling many common technical requirements of modern web development, but these claims are largely anecdotal. Furthermore, business applications tend to differ in their technical requirements compared to regular public facing web applications. Thus, it is unclear if SvelteKit would be a good fit for the development of modern business applications.

\section{Purpose of this Study}
\label{sec:purpose-of-this-study}
This study examines SvelteKit's claimed benefits in the domain of modern business applications. To this end we compare multiple implementations of the same business application in terms of performance, development effort, and code complexity. Furthermore, we provide further considerations when using SvelteKit in production.

In this study, we investigate how SvelteKit's claimed benefits translate to the development of business applications. To this end we reimplement parts of an application that is used to schedule rides for a chauffeur service provided by a federal agency. We compare the resulting applications with the original implementation in terms of performance and development effort. 

% Furthermore, we want to generalize the results from this project to typical business applications and try to provide a blueprint for future projects.

\pagebreak

\section{Thesis Structure}
This thesis is structured in the following way: \Cref{ch:fundamentals} will lay out fundamentals required for this thesis. This includes an overview of Svelte and SvelteKit, as well as SAP OpenUI5. In \Cref{ch:related-work} we will provide an overview of related research. Afterwards, \Cref{ch:implementation} will provide details about the specific code artifacts and their implementation. Following that, in \Cref{ch:evaluation} we will give our evaluation results. Finally, in \Cref{ch:discussion} we will discuss our results and provide directions for future work.


\chapter{Fundamentals}
\label{ch:fundamentals}

This Chapter introduces fundamental knowledge that will be required in the following thesis.


\section{Methodology}
\label{sec:methodology}

This thesis follows the principles of an explorative case study as defined by P. Runeson and M. Höst\cite{runeson_guidelines_2009}. The Authors describe the Process as consisting of five steps:

\begin{enumerate}
    \item Case study Design: Objectives are defined, and the case study is planned.
    \item Preparation for data collection: procedures and protocols for data collection are defined.
    \item Collecting evidence: execution with data collection on the studied case.
    \item Analysis of the collected data.
    \item Reporting
\end{enumerate}

As defined in \ref{sec:purpose-of-this-study}, this study intents to provide evidence that points towards or against SvelteKit's applicability for the development of business applications.

\subsection{Definition of Data Collection}
\begin{enumerate}
    \item Performance metrics
    \item complexity metrics
\end{enumerate}

\subsection{Collection of evidence}

Metrics applied to all implementations


\subsection{Evaluation}

Factual evaluation:
\begin{enumerate}
    \item LoC
    \item performance
    \item ecosystem size
\end{enumerate}

\section{Programmer Experience}


\section{Web Development}
\label{sec:web-development}

\subsection{History}

\subsection{Rendering Patterns}

\subsection{Meta Frameworks}

\section{Svelte}
\label{sec:svelte}
% - [ ] Svelte <span style="color: red">[important]</span> 14.8
% - [ ] What is Svelte
%     - [ ] origins
%     - [ ] created by Rich harris
%     - [ ] successor to Ractive.js
%     - [ ] most loved framework 2021
%     - [ ] compiler based approach
% - [ ] Features
%     - [ ] svelte files
%         - modified html
%         -  similar to vue3 because vue author inspired by svelte predecessor (source?)
%     - [ ] reactivity
%         - `$` to make statement reactive
%         - usual assignment operators to update
%         - possible because svelte uses compiler
%         - show example of generated code
%     - [ ] templating
%         - very simple
%         - if
%         - each
%     - [ ] events
%         - on:click
%     - [ ] bindings
%         -  bind:value
%     - [ ] components
%         - component syntax
%     - [ ] store
%     - [ ] scoped styling
%     - [ ] animations

% svelte is component library
% created by Rich Harris as successor to Ractive.js \cite{offerzen_origins_svelte_2022}
% 
% language for developing user interfaces on the web \cite{offerzen_origins_svelte_2022}
% 
% framework not something to run code but to think about code
% 
% tries to do as much as possible at build-time instead of runtime
% 
% compiler means it is possible to do things which would not work in vanilla JS

\section{Features}

\subsection{Svelte Files}
% - Files with .svelte ending
% - Syntax very similar to regular html with some extensions
% - standard javascript/Typescript $=>$ existing tools can be more easily integrated
% - javascript/html/css colocated in same file
% - {} to use js expressions in html, similar to react.
% - shorthand attribute syntax

\begin{minted}{html}
<script>
    let count = 2;
</script>

<div>count: {count}!</div>
\end{minted}

\subsection{Reactivity}
% - asignments with usual js asignments. would not work in React.
% - unlike react, svelte runs script just once, for assignments, svelte inserts invalidate blocks to update a variable
% - compare svelte to equivalent react code.

\begin{minted}[highlightlines={4-6,9}]{html}
<script>
    let count = 2;

    function increment() {
        count += 1;
    }
</script>

<button on:click="{increment}">click me!</button>
<div>count: {count}!</div>
\end{minted}

\begin{minted}{js}
function increment() {
    $$invalidate('count', count += 1)
}
\end{minted}

\begin{minted}{jsx}
function Component() {

    const [count, setCount] = useState(2);

    function increment() {
        setCount((c) => c + 1);
    }

    return <>
        <button onClick={increment}>click me!</button>
        <div>count: {count}!</div>
    </>
}
\end{minted}


\begin{minted}[highlightlines={4}]{html}
<script>
    let count = 2;

    $: doubled = count * 2;

    function increment() {
        count += 1;
    }
</script>

<button on:click="{increment}">click me!</button>
<div>count: {count}!</div>
<div>doubled: {doubled}!</div>
\end{minted}

% - $: to make reactive statement
% - repurpose of rarely used js feature
% - used for reactive assignments, conditions, and whole codeblocks
% - reactive block is rerun everytime a variable used inside the block changes

\subsection{Templates}

\begin{minted}[highlightlines={11-14}]{html}
<script>
    let count = 2;

    function increment() {
        count += 1;
    }
</script>

<button on:click="{increment}">click me!</button>
<div>count: {count}!</div>
{#if count > 9}
    <div>Vegeta what does the scouter say about his power level?</div>
    <div>It's over 9!!</div>
{/if}
\end{minted}


\subsection{Events}

\subsection{Bindings}

\subsection{Components}

\section{SvelteKit}
\label{sec:sveltekit}

In addition to Svelte's core functionality, the Svelte ecosystem has further expanded with the introduction of SvelteKit. Developed as an official framework for building web applications, SvelteKit builds upon the foundations of Svelte and provides additional features and tools that streamline the development process. By leveraging the power of Svelte's compilation-based approach, SvelteKit enhances developer productivity, scalability, and provides a cohesive framework for building robust web applications.

\subsection{Key Features of SvelteKit}

SvelteKit introduces several key features that extend the capabilities of Svelte and facilitate the development of complex web applications:

\subsubsection{Server-Side Rendering (SSR)}

One of the notable additions in SvelteKit is the built-in support for server-side rendering (SSR). SSR enables rendering components on the server, which can improve initial page load performance and enable search engine crawlers to index content more effectively. SvelteKit seamlessly handles the rendering process on the server, allowing developers to build dynamic and interactive applications while benefiting from the performance advantages of SSR.

\subsubsection{Routing and Navigation}

SvelteKit provides a powerful routing system that enables developers to define and manage application routes effortlessly. The framework supports both client-side and server-side routing, allowing for smooth navigation between pages while preserving SEO benefits through SSR. SvelteKit's routing capabilities include dynamic routes, nested routes, and route guards, empowering developers to create sophisticated navigation structures.

\subsubsection{Built-In Data Fetching}

Fetching and managing data is a fundamental aspect of web application development. SvelteKit simplifies this process by offering built-in data fetching capabilities. Developers can easily define data dependencies for each page or component, and SvelteKit handles the data retrieval and synchronization. This feature streamlines the development workflow, reduces boilerplate code, and ensures optimal data fetching strategies, such as parallel or sequential requests, based on the application's needs.

\subsubsection{Preloading and Prefetching}

To optimize the user experience, SvelteKit provides preloading and prefetching mechanisms. Preloading allows the framework to anticipate and fetch resources required for subsequent pages, reducing latency and improving perceived performance. Prefetching, on the other hand, enables developers to specify resources that should be fetched in the background, ensuring that subsequent interactions or navigation are seamless and responsive.

\subsubsection{Serverless Deployment and APIs}

SvelteKit embraces the serverless paradigm, enabling effortless deployment of applications to serverless platforms like Netlify, Vercel, or AWS Lambda. The framework facilitates the creation of serverless functions, allowing developers to define custom server-side endpoints and build APIs without the need for separate server-side code. This streamlined deployment and serverless functionality simplify the scaling and maintenance of web applications, making SvelteKit a suitable choice for modern cloud-native architectures.

\subsection{Advantages of SvelteKit}

By combining the power of Svelte's compilation-based approach with the additional features provided by SvelteKit, developers can benefit from a range of advantages:

Improved performance through server-side rendering and optimized data fetching strategies.
Enhanced development productivity with built-in routing, data management, and serverless deployment.
Simplified code organization and maintenance due to the cohesive framework architecture.
Seamless integration with the Svelte ecosystem, enabling developers to leverage existing Svelte components, libraries, and tools.
It is worth noting that, similar to Svelte, SvelteKit is a relatively new framework, and its ecosystem is still evolving. While it provides a solid foundation for building web applications, developers should consider the maturity of third-party integrations and community support when selecting SvelteKit for their projects.

In conclusion, SvelteKit expands upon the capabilities of Svelte, offering a comprehensive framework for building web applications. With features such as server-side rendering, routing, data fetching, and serverless deployment, SvelteKit empowers developers to create high-performing and scalable applications with an improved development experience. By leveraging the strengths of Svelte's compilation-based approach, SvelteKit enhances productivity and provides a robust foundation for modern web development.

\section{Business Application}
\label{sec:business-application}

Our applications usually not public facing $\rightarrow$ SEO not relevant $\rightarrow$ SSR, FCP, TTI not as relevant

\chapter{Related Work}

\begin{itemize}
    \item Angular and Svelte Frameworks: a Comparative Analysis \cite{tripon_angular_2021}
    \item DOM benchmark comparison of the front-end JavaScript frameworks \\React, Angular, Vue, and Svelte \cite{levlin_dom_2020}
\end{itemize}

\chapter{Implementation}

\section{Chauffeur Service}

%  \begin{itemize}
%     \item Take some Pictures
%     \item Chauffeur service for German federal agency.
%     \item Takes new Chauffeur Jobs
%     \item UI overview of Jobs, and Jobs Details
%     \item Overview of Chauffeurs and their availability 
%     \item UI to manage Jobs, assign chauffeurs
%     \item group Jobs
%  \end{itemize}

As a basis for our case study we used an application called DSW-FD (Dispositions-software-Fahrdienst). DSW-FD is used to schedule chauffeur rides for a German federal agency. The application provides a rich user interface, with views for managing chauffeur jobs, as well as chauffeur drivers. Clients which need a Chauffeur ride can call a separate hotline, where a handler creates a new chauffeur job, this job is then sent to DSW-FD where it appears in the overview of Jobs, and can then be processed by a Job handler.

Drivers need to be manually assigned to a job by a handling person. To this end, the application provides suggestions for drivers which would be free when the job is schedule and are closest to the job's departure point. Furthermore, the app provides functionality to group together different jobs so that they may be handled by a single driver, determine the return destination for the driver after they finished the job, as well as marking a job as being handled by a pool of drivers.

The application also provides features to directly interact with drivers, such as broadcasting messages to all drivers, reminding a driver to take their mandatory break, and directly calling a driver.

\section{Current Implementation}

\begin{figure}[ht]
    \centering
    \includesvg[width=.8\linewidth]{assets/dswfd-architecture}
    \caption{Root of the \textsc{CouchEdit} configuration metamodel}
    \label{fig:metamodel-base}
\end{figure}

\begin{itemize}
    \item UI5-Frontend
    \item Java-Backend
    \item Android app for drivers.
    \item OIDC
    \item mssql
\end{itemize}


\section{SvelteKit Implementation}

\begin{itemize}
    \item Two approaches (full stack, FE only)
    \item First try UI5 web components, second try CSS classes
    \item Authentication
\end{itemize}
\chapter{Evaluation}
\label{ch:evaluation}

\todo{Chapter intro}

\section{Performance}

\todo{Benchmark SvelteKit without SSR, maybe Next.js Benchmark}

We tested the existing UI5 implementation and the new SvelteKit Implementations, in multiple metrics. We were especially interested in responsiveness to navigation events. To this end we measured first contentful paint (FCP), the time until the page settled, and cumulative layout shift (CLS). Furthermore, we measured time until the page settled, for a navigation from the chauffeur job overview, to the details view of a chauffeur job \todo{This paragraph could use some work}.

All measurements were conducted on a Lenovo ThinkPad T480, with an Intel Core i7-8550U, and 16 GB RAM, running Windows 11. The measurements where taken using the developer tools of Google Chrome 116.0. The benchmark setup is described in app. \Cref{app:benchmark-setup}.

Our measurements (\Cref{fig:benchmark}) show, while the SvelteKit implementations take longer to send content to the client, they reduced time until the relevant data is being shown, significantly. These results align with the expected effects of server side rendering outlined in \Cref{sec:rendering-patterns}. Furthermore, SvelteKit shows significant improvements when navigating between two pages. This is improved even further by SvelteKit's preloading capabilities. In our benchmark, preloading was set to "tap", meaning, the data is being fetched as soon as a \mintinline{html}|mousedown| event is triggered.  

\begin{figure}
    \centering
    \begin{tikzpicture}
        \begin{axis}[
                ybar,
                bar width=.7cm,
                width=.9\textwidth,
                height=.5\textwidth,
                legend style={at={(0,1)},
                    anchor=north west,legend columns=1},
                legend cell align={left},
                symbolic x coords={TTFB,FCP,CP,Nav},
                xtick=data,
                enlarge x limits={0.15},
                nodes near coords,
                nodes near coords align={vertical},
                ymin=0,ymax=4000,
                ylabel={duration (ms)},
            ]
            \addplot table[x=type,y=ui5]{\benchmarkdata};
            \addplot table[x=type,y=sk-fs]{\benchmarkdata};
            \addplot table[x=type,y=sk-fe]{\benchmarkdata};
            \addplot table[x=type,y=sk-spa]{\benchmarkdata};
            \legend{UI5, SvelteKit Full Stack, SveleKit FE-only, SvelteKit SPA}
        \end{axis}
    \end{tikzpicture}
    \label{fig:benchmark}
    \caption{Benchmark Results for TTFB, FCP, CP and Nav}
\end{figure}

\section{Bundle Size}

\todo{SvelteKit is way smaller because of tree shaking and code splitting}

\section{Experience}

% \begin{itemize}
%     \item Both approaches (full stack and frontend only) worked great
%     \item advantage of full-stack: 
%     \item no API definition required
%     \item reuse of models between UI and business logic or even type inference
% \end{itemize}

In this section we give our personal experience developing with SvelteKit in an attempt to reason about SvelteKit's expected developer experience. As we noted in \Cref{sec:methodology}, providing a heuristic analysis of SvelteKit's DX is out of scope.

In our experience both discussed implementations (full stack in SvelteKit and SvelteKit frontend only) enabled an efficient and pleasant development experience. Especially the full stack approach significantly reduced overhead, because models could be reused between client and server and communication between server and client happened transparently.

% \begin{itemize}
%     \item Concise / Intuitive Syntax
%     \item reduction of client-server waterfalls
%     \item SPA/SSR/SSG with single framework
%     \item clear data flow (load function $\rightarrow$ svelte, svelte $\rightarrow$ action)
% \end{itemize}

\subsection{Project Setup}

SvelteKit provides a command line utility to create new projects (\Cref{fig:project-setup}). This utility provides options to configure TypeScript for type checking, ESLint\footnote{\url{https://eslint.org/}} for code linting, Prettier\footnote{\url{https://prettier.io/}} for code formatting, as well as Vitest\footnote{\url{https://vitest.dev/}} and Playwright\footnote{\url{https://playwright.dev/}} for testing. 


% \begin{itemize}
%     \item Fast, \mintinline{bash}|npm cgeate svelte@latest|
%     \item missing out-of-the-box features: form validation/parsing
% \end{itemize}

\begin{figure}
    \centering
    \includegraphics[width=.95\linewidth,trim={0 15cm 0 1.5cm},clip]{assets/sveltekit-project-setup}
    \caption{Process for creating a new SvelteKit project}
    \label{fig:project-setup}
\end{figure}

\subsection{Syntax}

We experienced Svelte's syntax as easy to grasp. This is because Svelte's syntax closely adheres to HTML with some added features for templating and reactivity.

Svelte's decision to reuse assignment operators for reactivity, not only makes for a less verbose, but also more intuitive syntax. Something, something code snippet:

\begin{myminted}{svelte}{Svelte}
<script>
    let count = 0;
</script>

<button on:click={() => (count++)}>+</button>
\end{myminted}

\begin{myminted}{jsx}{React}
import { useState } from 'react';

export function Component() {
    const [count, setCount] = useState(0);

    return <button onClick={() => setCount(c => c + 1)}>+</button>
}
\end{myminted}

This example clearly shows how Svelte simplifies a common use case, by utilizing its compiler based approach.

\todo{}

Simple Store API:
\begin{myminted}[escapeinside=||]{svelte}{}
<script>
    import { writable } from 'svelte/store';

    const todos = writable([]);
</script>

<button on:click={() => todos.}>add Todo</button>
{#each |\$|todos as todo}
    <div>{todo.text}</div>
{/each}
\end{myminted}

Simple Reactive Statements:
\begin{myminted}[escapeinside=||]{svelte}{}
<script>
    let count = 0;
    |\$|: doubled = count * 2;
</script>
\end{myminted}

% \subsection{Library Interoperability}

% \todo{}

% \begin{itemize}
%     \item Nice interop because Svelte is close to vanilla HTML/JS/CSS
%     \item Svelte Code is valid HTML/JS/CSS Syntax $\rightarrow$ Tools work out of the box (e.g Prettier, ESLint)
% \end{itemize}


\subsection{Routing}
SvelteKit decided to integrate a filesystem-based routing system. This comes with some inherit advantages and disadvantages. Probably the biggest advantages is, that this approach makes immediately clear what the purpose of a file is. Especially for simple cases this routing system works with no overhead and makes it possible to work very fast, when adding new routes. But some problems become apparent in more complex examples. Projects with a large amount of routes will inevitably have many files with the same name. In our experience this could make it harder sometimes to use IDE search functions to quickly navigate between many files, because searching most of the time returned more than one \mintinline{bash}|+page.svelte| or \mintinline{bash}|+page.js| file. Furthermore, solutions to some problems can cause a lot of file changes in a version control system. During our implementation efforts, we found the need to wrap multiple routes in a layout group. This meant, all files hat to be moved to a new subdirectory. This caused a lot of noise in the version control system for a change that was simple in nature. With a declarative routing solution the same problem could have been solved by simple changing the router configuration.

It has also to be noted that this filesystem-based approach to routing makes it effectively impossible to split a singular SvelteKit application into multiple smaller projects. Because routing is strictly bound to the filesytem, it is only possible to configure routing in the root project.

% \subsection{Clear data flow}

% Clear direction: +page.server.js $\rightarrow$ +page.js $\rightarrow$ +page.svelte.

\subsection{Svelte works best with plain HTML}
\label{sec:evaluation-ui-libs}

As described in \Cref{sec:implementation-ui}, the implementation was required to follow the SAP Fiori Design Guidelines. To this end we tried using both UI5 web components and a plain CSS classes approach using SAP Fundamental Styles. In either case we decided to create a library of Svelte components that wraps the actual UI utility. This decision was made to reduce the amount of boilerplate code and repetition required to use some UI elements. Furthermore, this decision made it easier to migrate from UI5 web components to fundamental styles, because only the Component library had to be changed. 

\todo{Finish section}

% In our implementation we experimented with different approaches to structuring our UI-code

% \begin{itemize}
%     \item Most directives do not work on Components (use:, class:, style: )
%     \item Scoped styles means, it is annoying to pass down locally defined CSS-classes to children.
%     \item But, Tailwind CSS can make things better (because everything is defined globally anyway).
% \end{itemize}


% \subsection{Cases where Svelte is bad}
% While SvelteKit can provide an improved development experience, it does not come without its flaws. During out study we identified multiple areas where SvelteKit caused problems, or could become a paint point in larger projects.

% \begin{itemize}
%     \item Centralized load function (compared to RSC's, load where needed)
%     \item Things Which require node.js server (e.g. cron, websockets)
%     \item Big data sets, which need to be shared between routes (virtuelle-fabrik)
%     \item Style is scoped per default, thus it is annoying to pass classes to child component (improved with tailwind, because everything is global)
%     \item Differences between client renderer and server renderer
%     \item No best practices yet
%     \item Problems with Svelte's Custom fetch
% \end{itemize}

\subsection{Universal vs. Server Load Functions}
\label{sec:evaluation-universal}

As noted in \Cref{sec:implementation-redirect}, we encountered multiple problems with the frontend only implementation that utilizes universal. The primary problem is, that SvelteKit provides no help for sending data to the backend. While data loading is streamlined through universal load functions. Server actions, as used in the full stack implementation, can only be used server side. This means, that submitting data has to be handled manually. Furthermore, this means that submitting data will only work when JavaScript is enabled. This is likely the main reason why SvelteKit provides no support for this use case, because prioritizes progressive enhancement.

% \begin{myminted}{svelte}{}
% <script>
%   async function handleSubmit({cancel, request}) {
%     cancel();

%     const res = await fetch('api/update', {
%         method: 'POST',
%         body: request.body
%     });

%     if (!res.ok) {
%       // error handling...
%     }

%     return ({ update }) => {
%       update();
%     }
%   }
% <script>

% <form method="POST" use:enhance={handleSubmit}>
%   <button type="submit">send</button>
% </form>
% \end{myminted}

% During this approach we also realized further complications. Firstly, authentication with the backend API becomes more complex, because both client and server need the access token for the API.
% could this be fixed by passing through the api token??

Furthermore, using fetch in universal load functions, requires usage of a special fetch function supplied by SvelteKit. This special fetch function makes sure that fetching data behaves the ame way on the client and server. Additionally, When a page with a universal load function is accessed, the load function is first run on the server during SSR, and the result is sent to the client. During hydration on the client side, the load function is then run again. This means, that ordinarily all fetch calls executed in the load function would run two times, once on the server and once on the client. To prevent this, SvelteKit's implementation inlines fetch responses on the server side, and then uses the inlined response on the client during hydration. This means, that it is necessary to use SvelteKit's fetch implementation. But, as this special fetch function is only exposed as an argument in load functions, one has to always pass this fetch function on any delegates, which require network communication. Furthermore, it is not possible to use custom HTTP-clients, such as Axios\footnote{\url{https://axios-http.com/}}.

% Svelte's custom fetch function further increased complexity. SvelteKit's implementation of fetch makes sure that fetch behaves similarly on the client and server. Therefore, it enables relative URL paths on the server, which would ordinarily not be possible.

% - problems with universal
% - SvelteKit fetch weird
% - auth more complicated
% - 

\subsection{Reactivity Pitfalls}

While Svelte's reactivity is overall very intuitive, it nonetheless has some pitfalls.

Svelte's reactive blocks work by analyzing all local variables that are used in this block. If one of those variables changes, the reactive block is rerun. But, this only works if the dependent variable is used directly in the reactive block. In cases where the dependency is used in a function for example, the dependency cannot be analyzed and therefore will not trigger executions of the reactive block:

\begin{myminted}[escapeinside=||, autogobble]{svelte}{}
<script>
    let count = 1;

    let doubled;
    |\$|: calcDoubled();

    function calcDoubled() {
    doubled = count * 2;
    }
</script>

<button on:click={() => (count++)}>+</button>
<div>{count} * 2 = {doubled}</div>
\end{myminted}

In this example, the function \mintinline{svelte}|calcDoubled|, calculates the value of \mintinline{svelte}|double| in correspondence to \mintinline{svelte}|count|. But because \mintinline{svelte}|count| is not directly used in the reactive statement, the reactive statement will not trigger when \mintinline{svelte}|count| is updated.

\subsection{Centralized load function}

SvelteKit's architecture enforces that all asynchronous data has to be acquired inside a pages load function, for it to be available during server side rendering. This provides a clean and easy to understand flow for simple implementations. But it can increase complexity in cases where a single isolated component has to be reused in multiple different routes.

One could for example imagine a simple component which displays a Twitter feed, which has to be used in multiple places. Given following file hierarchy:

\begin{myminted}{html}{}
routes/
  a/
    +page.js
    +page.svelte
  b/
    +page.js
    +page.svelte
  TwitterComponent.svelte
\end{myminted}

Both route a and b need to use the twitter component. The way to implement this in SvelteKit would be to fetch the data in the load function of both routes and pass it to the twitter component as a prop:

\begin{myminted}{js}{\string{a,b\string}/+page.js}
export async function load({ fetch }) {

    const twitterFeed = (await fetch('/api/twitterfeed')).json();

    return {
        twitterFeed,
        // other page data...
    }
}

\end{myminted}

\begin{myminted}{svelte}{\string{a,b\string}/+page.svelte}
<script>
    import TwitterComponent from '../TwitterComponent.svelte';

    export let data;
</script>

<TwitterComponent feed={data.twitterFeed} />
<!-- other page markup... -->
\end{myminted}

\begin{myminted}{svelte}{TwitterComponent.svelte}
<script>
    export let feed
</script>

{#each feed as tweet}
    <!-- ... -->
{/each}
\end{myminted}

This pattern needs to be repeated for each route that wants to use the twitter component. This not only results in unnecessary code duplication, but can also cause problems later in a project's lifecycle. If the twitter component is to be removed later on, it is imaginable that one of the fetch calls is left in by accident, because the data fetching is detached from where the data is actually used, thus causing unnecessary traffic.

Other approaches circumvent this problem by coupling data fetching more closely to its usage. For example, a similar implementation using React Server Components could look something like this:

\begin{myminted}{jsx}{TwitterComponent.jsx}
export async function TwitterComponent() {
    const twitterFeed = await (await fetch('/api/twitterfeed')).json();

    return <>
        {twitterFeed.map(tweet => (
            // ...
        ))}
    </>
}
\end{myminted}

\begin{myminted}{jsx}{\string{a,b\string}/page.jsx}
import TwitterComponent from '../TwitterComponent';

export async function Page() {
    return <>
        <TwitterComponent />
        { /* other page markup... */ }
    </>
}
\end{myminted}

With this approach, the Twitter component becomes truly isolated from the page, because it can fetch its own data. We expect this architecture to scale better in large applications, than SvelteKit's centralized load function.


\subsection{Deployment}

SvelteKit provides first-class adapters that automate creating production deployments for various hosting providers such as Vercel\footnote{\url{https://vercel.com/}}, Netlify\footnote{\url{https://www.netlify.com/}}, Cloudflare Pages\footnote{\url{https://pages.cloudflare.com/}}, and Cloudflare Workers\footnote{\url{https://workers.cloudflare.com/}}. Furthermore, adapters exist to create standalone NodeJS server, or generate a static bundle that can be served from a simple file-base web server. Beyond this, a wide range of community plugins exist which further extend the range of platforms, SvelteKit can be deployed to. This means that SvelteKit generally supports great flexibility for deployment with comparatively little effort.  

But, this wide range of supported platforms creates some shortcomings. SvelteKit does mostly not implement features that require platform specific functionality. One such feature would be a hook that is executed when the application shuts down. This can sometimes be required for example to close open connections. While this can be easily worked around in a NodeJS environment, by using \mintinline{js}|process.on('sigint', ...)|, other features are not added this easily. 

Websockets are another feature that is currently not supported out of the box, because some providers such as Vercel have no support for websockets yet \cite{noauthor_vercel_nodate}. While it is possible to add websocket support to the NodeJS platform this is not possible without shortcomings. By default, the NodeJS adapter creates an entry point that starts a web server to host all handlers required by SvelteKit. It is possible to forego this entry point and instead write a custom server that uses the handlers directly. In this way it is possible to create a web server that uses websockets. But, this approach only works for the production build, in development, another workaround is required that works with hot-module replacement. Furthermore, a separate 

\todo{SvelteKit wants to be a serverless framework\cite{noauthor_expose_nodate}}

\subsection{Stability}

% \begin{itemize}
%     \item Svelte 2 to 3 was big change
%     \item Svelte 5, while being maintained for 2 major versions, will also come with migrations.
% \end{itemize}

Version 1.0 of SvelteKit released only recently in December 2023 \cite{team_announcing_2022}, thus it is not possible to tell how stable the APIs of the framework will be going forward. Svelte on the other hand has reached its fourth major version. So far, version 3 made changes to its API, which caused a need for manual migrations. 

\todo{Finish section}


\subsection{Future of Svelte}

On 20. September 2023, The Svelte team gave a first preview of the next major new API planned for version 5 of Svelte, called runes. Runes are a set of compiler instructions that replace the system of reactivity currently used in Svelte. With runes, Reactivity becomes more explicit. Instead of every top level variable in a svelte file being reactive by default, state that should be reactive has to be marked with the \mintinline{svelte}|$state| rune:

\begin{myminted}[escapeinside=||]{svelte}{}
<script>
    let count = |\$|state(0);
</script>
\end{myminted}

While initially being more verbose, this has multiple advantages. Firstly, this syntax is more explicit, clearly marking which variable is reactive and which is not. Secondly, this syntax is not limited to the top-level of a svelte file, further, it will even be possible to place state-runes in JavaScript files. This will make it easier to move out state in to separate files. Where as before, stores were required to place state in separate files, with this new syntax it will be possible to use the same syntax in svelte-files and in js-files.

Furthermore, Svelte 5 introduces two runes that replace the \mintinline{svelte}|$:| syntax, \mintinline{svelte}|$derived| and \mintinline{svelte}|$effect|:

\begin{myminted}[escapeinside=||]{svelte}{}
<script>
    let count = |\$|state(0);

    const doubled = |\$|derived(count * 2);

    |\$|effect(() => {
        console.log(`new value of count: ${count}`);
    })    
</script>
\end{myminted}
\s{$}

\todo{Finish section}

% svelte likely wont go away
\chapter{Conclusion}
\label{ch:conclusion}

\todo{Write whole chapter}

\appendix
% hier Anhänge einbinden
\input{chapters/sources}

\backmatter
% \nocite{Knappen2009}
% \nocite{Mittelbach2005}
% \nocite{Schlosser2014}
% \nocite{Sturm2012}
% \nocite{Voss2010}

\printbibliography

\clearpage
\thispagestyle{empty}

Name: \fullname \hfill Matrikelnummer: \matnr \vspace{2cm}

\minisec{Erklärung}

Ich erkläre, dass ich die Arbeit selbständig verfasst und keine anderen als die angegebenen Quellen und Hilfsmittel verwendet habe.\vspace{2cm}

Ulm, den \dotfill

\hspace{10cm} {\footnotesize \fullname}
\end{document}

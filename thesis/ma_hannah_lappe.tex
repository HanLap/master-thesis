\documentclass[
a4paper, 
DIV=15, 
headsepline, 
numbers=noenddot,
bibliography=totoc,
BCOR=15mm,
leqno,
]{scrarticle}

\usepackage[scaled]{helvet}

\title{Development and Evaluation of a Technology Stack for Business Applications using SvelteKit}
\author{Hannah Lappe}
\date{\today}


\begin{document}

\maketitle

\section*{Abstract}
% Over the past decade, single page applications (SPAs) have been the primary approach for developing websites.  SPAs can provide feature rich software directly through the browser. But classic SPAs tend to lack in certain aspects. Metrics such as time to interactive (TTI) usually suffer because traditional SPAs have huge JavaScript bundles that need to be sent to the web browser. Recent years have seen a trend of multipage applications becoming more Popular again. This trend is enabled by modern web frameworks such as Next.js, Gatsby and SvelteKit. 

% Svelte is an up-and-coming JavaScript UI Framework that has gained a lot of popularity in recent years. Contrary to established frameworks, Svelte uses code compilation to produce a more efficient JavaScript bundle, thus vastly improving performance.

% SvelteKit is a framework that builds around Svelte and provides common features such as app routing, server side rendering (SSR), and data fetching. SvelteKit is a full stack framework and thus can also be used to handle 

Over the past decade, single page applications (SPAs) have become the primary approach for developing websites. This development was fueled by JavaScript frameworks such as React, Angular and Vue which made it easy to develop rich web applications. But SPAs tend to suffer in certain metrics, such as initial load times and responsiveness. To amend these issues, recent trends show a return back to a server side rendering (SSR) approach. This transition is enabled by the emergence of so-called meta frameworks. Meta frameworks provide many of the tools required for building modern web applications out of the box. Thus, developers need to spend less time thinking about software architecture and can spend more time thinking about technical requirements.

SvelteKit is an up-and-coming JavaScript meta framework that is rapidly gaining popularity. It builds around Svelte, a new competitor to web UI frameworks. SvelteKit claims to handle many of the best practices for modern web applications such as build optimization, offline support, preloading and SSR. Furthermore, as a meta framework SvelteKit makes it possible to develop frontend and backend in the same code base.

In this work, we investigate if SvelteKit's claimed benefits translate to the development of business to business (B2B) applications. To this end we analyze typical requirements of business applications and provide a blueprint that can be used to develop new business applications using SvelteKit. Furthermore, we demonstrate the feasibility of this blueprint on a real world example. Finally, we compare our solution to a more traditional software stack that leverages Java and React.

% In this work, we investigate how SvelteKit can be applied to the development of business applications. To this end, we analyze typical requirements of business to business (B2B) applications and develop a blueprint that can be applied to common business application use cases. We demonstrate this blueprint on a real world scenario. Finally, we compare SvelteKit to a more traditional software stack that leverages Java and React.



Der Fahrdienst für eine Bundesbehörde mit rund 150.000 Chauffeurfahrten pro Jahr soll durch eine entsprechende IT-Unterstützung reibungslos und effektiv durchgeführt werden.
Unter Einhaltung höchster Sicherheitsanforderungen wurde eine eigenständige Applikation entwickelt. Anhand von Dienstplänen, die mit SAP-HCM synchronisiert werden, erfolgt eine automatische Fahrzeugzuordnung der Chauffeure und eine Disposition der Fahrten. Die Chauffeure erhalten über eine Android Applikation ihre Dienstzeiten und anstehende Fahrten werden per "Notifications" signalisiert. Fahrten und Schäden werden mobil erfasst und umgehend in den Backend Systemen weiterverarbeitet. Die Aus- und Rückgabe der Fahrzeugschlüssel erfolgt über, in den digitalen Prozess integrierte, Schlüsseltresore.




\end{document}